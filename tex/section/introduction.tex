\section{Introduction}

In the History of the Church, there has throughout the centuries been a debate
about what sin is and what is not sin. Therefore many church historians have tried to 
define what the complete list of sins is. In this document, this has been 
explicitly studied in the Greek. It began with a thorough reading of the New
Testament to find all the lists of vices presented at least in clusters of three.
The lesser lists have been left out. Thereby each verse or passage containing a
set of vices has been compiled into a long comprehensive list. From that, they
have been sorted, and words of the same lexeme have been considered the same vice.
In some cases, a vice is described with a multiword phrase, otherwise, the vices
are always described with one word with the intent to understand them from a
pure NT perspective and to be understood by regular bible readers as well as 
clergy.

While a bible reader and a Christian believer can read and understand sins 
from a context, it is not always sure they can interpret and understand one-word vices.
The main problem lies within the prestudies before making a new bible translation.
Too much trust is given to the sources or dictionaries that describe the 
semantic values of a specific word without giving a deeper insight into its proper
semantic value and understanding from a secular ancient use. Therefore it is 
very easy for a bible translator to do an appropriate guesswork of what it actually
should mean but this guess is also affected by the religious views and dogmas, or
disbelief of the translator. Often there are also directives that affect how
the content of the bible should be read or portrayed to the reader, and therefore
the train of thought from ideologies also colors the translation and misrepresents
the word of God.

When a reader tries to understand the sins of NT, there is confusion about what actually
is meant by the translator, very often translations lose semantic value in the 
translation process and the result is a vague translation. When believers try to conduct
hermeneutical studies it is sometimes hard to fully understand the true meaning and
religious interpretation becomes dull. Because of dullness, a bland religious understanding
is developed which leads to doctrinal use that is off-track with the true Gospel.
When a term is translated among lists of vices the result is a bland understanding
of its true meaning. In a church setting where the collective understanding becomes vague
because of dull translation when a passage can not be fully understood or the meaning of a 
word or phrase, there is too much room for emotional interpretation, that leads to the 
incubation of a culture that becomes off-track or even detrimental to the assembly of the 
believers. 

The purpose of this study is to restore the true meaning of words that count as vices
in the New Testament, to list them, and to study them systematically to try to fully 
understand their true biblical meaning. And if possible give a deeper understanding of what
truly is sin and not. Many topics have been studied and lists are obviously connected to
certain circumstances and also have different consequences. Deep understanding, which
presumably could be found such as whole groups of topical vices throughout the
whole NT has been left out of this scope. There have been many things I wished I could 
have studied deeper but because of my limited time, someone else has to continue going 
deeper in this.

The document has been authored to be used as a dictionary itself, with a well-formulated understanding
of vices. A full set of verses and indices with topical appendices have been put last.
