\section{Introduction}

Throughout the history of the Antiquity and the Church, there have been lots of
philosophical debates regarding what is considered moral degradation of a person 
but also of society. What is right and wrong has been debated by the philosophers, 
by the church fathers, and appointed by prophets to people from God. Throughout 
history, thinkers and clergymen have compiled lists of virtues and vices. Some 
famous works are the Decalogue, ethics by Socrates, Plato, and Aristotle, the 
Stoics, the Didache, and Dante's seven deadly vices, to mention a few examples.

In the Judeo-Christian context, right and wrong are described in terms of 
commandments and sins, since it was given first to the Israelites through Moses 
the prophet. Later, Jesus came along as the Messiah -- as believed by Gentile 
confessors -- to reform Judaism. Right after that, the faith was spread by the 
apostles to the nations and became adopted as Christianity by the common public. 
When the prophets and apostles wrote the scriptures -- which were fully canonized 
into the Bible as late as 367 A.D. -- it became the world's most authoritative 
work of right and wrong. As the most influential religion, believers sought to
comply to their greatest extent, sadly power-hungry institutions that wanted control
adapted to the Christian faith with bias. From there on the scriptures has been
abused as a mean to manipulate populations with the threat of eternal damnation
if they did not comply with whatever certain biased interpretation.

Because of the abuse of holy scriptures over time in history for private gain,
it is of utmost importance to settle the truth of the Bible when it comes to the 
vices so that illegitimate use gets disdained and proper use regained. 

Since the Reformation, it has become common to translate the Bible into the 
language of the everyday reader. That's why lots of people have received a 
personal faith in the Christian God as practiced by protestants and some others. 
This leaves a lot of room for private interpretation. That which happens, when 
clergy reads the Bible without properly studying it, is that they perform 
according to the expectations of their denomination's consensus, which usually 
is based on dogmas and earlier developed traditions that put a bias in the practice 
of preaching. Most traditions which are based on Lutheran theology mix the vices 
of the Old and the New Testament. According to the Bible itself, it is not fit 
to teach commandments of men or Jewish myths as Paul, the apostle mentions in 
Tit 1:11--14, also the legalistic teaching of mosaic law is not necessary for 
the Christian Gentiles to practice, as written in Gal 5:4 and v.18.

For a believer to understand the sins in the New Testament, they usually read
whole passages in a context that explains what a specific sin is. The problem is
that there are complete lists of vices in the NT that are not described with 
a context but use only a phrase of at most one word to give its
meaning. Mostly, when Bible translations are conducted, by a small team of knowledgeable 
people, they use religious Greek dictionaries, which have a narrow semantic value 
describing the Greek words. The entries of those words are usually adapted for 
use in a Christian religious context. Therefore, those dictionaries don't reflect
the true meaning in a secular ancient context. In the long run, this has the effect
of a religious echo chamber that over time, shifts the understanding of the
meaning. Then, when the translators try to understand how they should translate,
it is hard to reconstruct a precise translation. Another thing also affecting
is budget and time limits. Therefore, many translations use a vague vocabulary
and produce a bland reconstruction of the meaning of the original manuscript.
The following problem of an ordinary reader is to grasp the true meaning of the
corpus when studying the Bible and thereby receive a vague message. When those
messages are understood, it is not truly clear what the intended communication of 
an original passage is. Later, building a religious practice out of such vague 
understanding often leads to guessing how it applies. This is often a very emotional 
interpretation that incubates the religious culture within an assembly of believers.
Sadly, when it comes to translating vices mentioned with one word, too much semantic
value is lost in the interpretation, which makes it impossible to understand the definition
of a vice. If the culture that has incubated in a religious group,
based on emotional understanding, has developed a toxic culture, it has
detrimental consequences for individual practitioners being accused wrongly of a
certain sin. In the end, the whole religious assembly may deteriorate, or turn into
a destructive cult.

The purpose of this study is to produce a complete list of vices that is a compilation of
all NT lists of sins written by the apostles. The restoration of the true meaning of the
words that count as vices in the New Testament, to list them, and to study them systematically, 
to fully understand their true biblical meaning. And if possible, give a deeper understanding 
of what actually counts as sin. The goal is to produce a dictionary, not only for academic purposes
but mainly for the purpose of Bible-based practiced religion for clergy, as well as laymen.
