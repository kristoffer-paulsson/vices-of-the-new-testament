\section{Introduction}

This report is about a study on the List of Vices in the New Testament, it has been a comprehensive study including renowned works in ancient Greek and biblical theology. Among others, LSJ is mostly from BW10 but also from https://perseus.tufts.edu for deeper understanding. It has to be mentioned that BW10 has reduced entries from renowned dictionaries, presumably censored. Left is the Thayer's dictionary and Gingrich and Dankers editions of BDAG, also accessed from BW10. Beyond that, the TDNT has been used extensively, and where satisfactory, it is the only used source. In some cases, neither a compound understanding of LSJ, BDAG, Thayer, or TDNT has truly been satisfactory. In some cases, an entire cognate study has been performed in the LJS only from Perseus in order to get a basic heathen understanding of the secular Greek meaning. In a few cases beyond that, even a word study from classical Greek works from antiquity has been conducted in order to truly secure a fundamental understanding. In several cases, a smaller hermeneutical study has been conducted in both the NA28 and LXX.

% When citing TDNT, Liddell or Thayer it has been decided to within the quotation to use transliterated greek in italics, in citation bible references % are adapted to SBLHS, and longer Greek quotations have been Romanized for those who are not scholars. The vices described in the entries in the this % report are compiled from: Gal 5:19--21; Matt 15:19; Mark 7:21; Rom 1:29--31, 13:13; 1~Tim 1:9--10, 2:8--10, 3:3--11, 3:13; Eph 4:31, 5:3--5; 2~Tim % 3:2--4; 1~Cor 5:10--11, 6:9--10; 2~Cor 12:20--21; Titus 1:7, 2:3, 3:3, 3:9; Col 3:5, 3:8--9; Heb 12:15, 12:16, 13:4; 1~Pet 2:1, 4:3; Acts 15:20, 15:29, 21:25; Rev 9:20--21, 21:8, 21:27, 22:15.
% The bible references inside the citations has also been adapted to the SBL Style Handbook.
% Also, the word \grc{πάροινος} is incorrectly cited by BibleWorks 10 compared to the real Liddell \& Scott.
% 
% Totally 126 unique original sins and not less than X ....

