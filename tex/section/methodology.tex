\section{Methodology}

When performing this study of vices, several approaches have been used
when trying to restore the true meaning of a certain Greek word or phrase.

The important thing is to be able to cite renowned works of other people,
historians, theologians, and so on. Because of this, only the most trustworthy
and respectable works have been consulted.

The most useful tool for effective studies has been the use of the bible software
BibleWorks v. 10 which has allowed systematic word and phrase studies with support
for morphosyntax. This has enabled searches for words based on their lexeme in which
the inflections have been tagged and made available in the search query.
 
Among others, Liddell \& Scott is used, mostly from BW10 but also from the
website https://perseus.tufts.edu for a deeper understanding. In some cases,
it's clear that the BW10 module for built-in LSJ has been selectively censored,
especially sensitive words in the presumably sexual arena.

The bible dictionaries known as Thayer's but also Gingrich and Danker's edition
of BDAG have been used. Also, The Theological Dictionary of the New Testament is 
heavily consulted and also authoritative.

First, the word has been looked up in Liddell to get a secular understanding,
then this has been compared to with Thayer's, Gingrich and Danker's dictionaries.
Sometimes this has been satisfactory but not always. In other cases, TDNT
has been researched which offers an extensive explanation of certain words and
phrases in all knowable aspects. If the TDNT has been enough it is the only source used.

In some special cases, cognate studies have been made. This means that all
the similar words with their semantic values have been added up to get a
better view of the true meaning of a Greek word. In even rarer cases most
of the cited ancient Greek resources have been looked up in their ancient
Greek context together with their English translation. Then those texts
have been understood and compounded into a certain understanding of a word
that was bland in the other dictionaries. A fine example is the research done
for the Greek word \grc{ἀσέλγεια}.

If the ancient Greek sources with their respective translations haven't produced
an interpretable result, then a word study has been performed in both LXX and NA28.

Hopefully, the methodology presented so far should have produced a result of 
entries over vices, of such good quality so that they may be used for further research
and also for religious use in Christianity, which is my goal and desire, to clean
up faulty chastisements that have hurt believers when the word of God has been
misinterpreted or used carelessly!
