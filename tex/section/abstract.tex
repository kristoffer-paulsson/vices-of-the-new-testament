\section*{Abstract}

In the History of the Church, there has been a debate about what sin is and what is not sin.
It has been detrimental for the believer not to fully know what those sins are. Mostly this
is because of vague bible translation which has led to a bland understanding of phrases for a certain vice.
This has given lots of room for private interpretation, mostly such interpretation that is 
collectively and emotionally incubated, which has hurt believers that fell victim to faulty
doctrines with a harsh judgment that followed. Many sloppy preachers have been laying out
doctrines in a misinterpreted way over time. 
In this study, the New Testament has been carefully read and all passages having a list of
at least three vices have been selected and sorted. Then, comprehensive studies have been
conducted in the ancient Greek language in combination with hermeneutical research to find a
true practical application. The big problem is that religion worked as an echo chamber, and
the true meaning of the Greek words used in the bible has shifted over the years since the Reformation.
Renowned theological dictionaries have been consulted, among
others TDNT, BDAG, and LSJ, mostly to try to understand the underlying secular use of a
word or phrase. In total 126 words and phrases were found, and after comprehensive studies, about 159 entries were authored.



