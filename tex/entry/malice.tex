\item[Malice,]
\entlbl{malice} 

\grc{κακοήθεια}
\index[grc]{κακοηθεια@\grc{κακοήθεια}} 
(\textit{kakoētheia}):
\newglossaryentry{kakoētheia}
{
    name=\grc{κακοήθεια},
    description={\entrefgls{malice}},
    sort=κακοηθεια@\grc{κακοήθεια}
}
According to Grundmann, the current term denotes ``It always means `wickedness,' `malice.' In the NT it occurs only at Rom 1:29 \ldots The series shows that it is here conscious and intentional wickedness.''\bkfoot{\grc{κακοήθεια}}{3:485}{\tdntGrundmann} 
Technically it is the intention to act in malignity towards  someone. \emph{Malice} defines as ``the intention to do something wrong and esp. to cause injury.''\cdfoot{malice}{2023-03-19}
See \entref{maleficent contriver}, \entref{maleficent craving}, \entref{maleficent eye}, also \entref{malevolence}.
Found in Rom 1:29.
