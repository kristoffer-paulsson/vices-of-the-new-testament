\item[Wine romp,]
\entlbl{wine romp} 

\grc{πάροινος}
\index[grc]{παροινος@\grc{πάροινος}} 
(\textit{paroinos}):
According to Thayer and Gingrich, both mention \emph{drunken} and \emph{wine addiction}. However, this is hard to believe because the average Christian is not supposed to fall into \emph{drunkenness}. It's more probable that the Bishop should be soberer than the average believer in Christ. Then Liddell mentions \grc{παροινικός} as ``addicted to wine, drunken'' but there is a whole array of cognates that assert a different behavior, \grc{πάροινος} ``\emph{befitting a drinking party}, \emph{drinking songs},'' \grc{παροινέω} \emph{play drunken tricks}, \grc{παροίνημα} \emph{drunkard's butt}, and \grc{παροινία} \emph{drunken behavior}. However, it seems this is related to imitating frolicsome drunken behavior rather than regular intoxication. Technically this is about a party with happy fellows and an unknown amount of wine. \emph{Romp} defines as ``a funny, energetic, and often sexual entertainment or situation.''\cdfoot{romp}{2023-03-24}
Found in 1~Tim 3:3; Titus 1:7.
