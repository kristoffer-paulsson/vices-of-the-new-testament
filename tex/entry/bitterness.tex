\item[Bitterness,]
\entlbl{bitterness}

\grc{πικρία}
\index[grc]{πικρια@\grc{πικρία}}
(\textit{pikria}):
\newglossaryentry{pikria}
{
    name=\grc{πικρία},
    description={\entrefgls{bitterness}},
    sort=πικρια@\grc{πικρία}
}
According to Michaelis, the term denotes ``originally `pointed,' `sharp,' e.g., of arrows \ldots, then more generally of what is `sharp' or `penetrating' to the senses, a pervasive smell \ldots, `shrill' of a noise \ldots, `painful' to the feelings \ldots esp. `bitter,' `sharp' to the taste \ldots The final sense, so also in derivatives like \emph{pikria}, `bitterness,' the bitter taste of plants \ldots, exerts the greatest influence; it is behind the transf. use in the sphere of the soul, where the experience of what is unpleasant, unexpected, or undesired is predominant. Thus the adj. is used with \emph{lypē} because it is painful, with laughter when this is tormented \ldots, with tears \ldots, not because they taste bitter, but because there is no desire to weep, because weeping bring no release.''\bkfoot{\grc{πικρία}}{6:122}{\tdntMichaelis{}}
Later in NT context M. connotes ``In Eph 4:31 \emph{pikria} stands at the head of a short list of vices. It is followed immediately by \emph{thymos} [\entref{outburst}] and \emph{orgē} [\entref{rage}]. Hence it does not mean the 'embitterment' which involves withdrawal and isolation (the word is not attested elsewhere in this sense) but `bitterness,' `resentment,' `an incensed and angry attitude of mind' to once neighbor.''\bksfoot{6:125}{Michaelis}
Clearly this bitterness is not avoidant but outgoing towards those around oneself, therefore it could also be \emph{sterness} which defines as ``the quality of being severe, or of showing disapproval.''\cdfoot{sterness}{2023-03-24}
Found in Eph 4:31; Heb 12:15.

