\item[Gossip,]
\entlbl{gossip} 

\grc{ψιθυρισμός}
\index[grc]{ψιθυρισμος@\grc{ψιθυρισμός}} 
(\textit{psithyrismos}):
Liddell connotes ``a whispering, \ldots whispering, slandering,'' then Thayer says ``(\ldots to whisper, speak into one's ear), a whispering, i. e. secret slandering,'' then Gingrich mentions ``whispering, gossip, talebearing,'' finally, Danker denotes ```information conveyed in a hushed tone', w. connotation of being denigrating covert gossip, tale-bearing.'' Denoting what D. says, the term is about slandering in secrecy and spreading rumors or illicit information, thereby the gravest slander of all, the one that is most important to set a stop to before people may wish to leave earth. \emph{Gossip} defines as ``conversation or reports about other people's private lives that might be unkind, disapproving, or not true.''\cdfoot{gossip}{2023-03-29}
Found in 2~Cor 12:20.
