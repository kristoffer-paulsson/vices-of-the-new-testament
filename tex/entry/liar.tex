\item[Liar,]
\entlbl{liar} 

\grc{ψεύστης}
\index[grc]{ψευστης@\grc{ψεύστης}} 
(\textit{pseustēs}):
\newglossaryentry{pseustēs}
{
    name=\grc{ψεύστης},
    description={\entrefgls{liar}},
    sort=ψευστης@\grc{ψεύστης}
}
In John 14:6, we need to know that Jesus claims to be the truth with capital T. In John 8:44, The Devil is the father of lies with capital L. Technically, the term \grc{ψεύστης} is used only 10 times in the NT, mostly by John and secondly by Paul. It is only used in contrast between the truth of God and the lies of the Devil. All truthfulness is in regard to the Kingdom of God and Jesus with the Gospel. 
Conzelmann denotes the correct use of \grc{ψεύστης} in profane greek as \emph{liar}. But the meaning thereof is far away from protestant and catholic state religious connotation, instead, C. describes ``[A.1.f] Lying cannot be viewed merely as the opposite of truth. Basic to the general and philosophical use of the word group is the twofold sense, namely, objective and subjective appearance, untruth as non-breaking and error as false judgment of reality. The norm of the ethical assessment of lying is the firm bond between \emph{alētheia} and \emph{dikē}. At issue is the divinely protected order of the world.'' 
Further, C. states ``[A.2] Hence the worst lie is perjury \ldots Subj. values enter in here. Lying esp. direct lying against others i.e., `calumniation' \emph{diabole} [\entref{she-devil}], is alien to the good man \ldots It deprives honor and is an assault on human dignity \ldots The aristocratic order demands esp. that one should not deceive those to whom one owes respect, \ldots On the other hand the gods deceive (\emph{dolos}) men.''\bkfoot{\grc{ψεύστης}}{9:595--6}{\tdntConzelmann{}}
\emph{Lie} defines as ``to say or write something that is not true in order to deceive someone,''\cdfoot{lie}{2023-03-10} and \emph{liar} as ``someone who tells lies.''\cdfoot{liar}{2023-03-10} When speaking against the truth, it is applicable to fall away from the Grace of God into the mosaic jurisdiction if they do not stop it.
Found in 1~Tim 1:10.
