\item[Irreverent,]
\entlbl{irreverent}

\grc{ἀσεβής}
\index[grc]{ασεβης@\grc{ἀσεβής}} 
(\textit{asebēs}):
According to Foerster (TDNT 7:185--7), for the pagans, irreverence meant ``In Athenian trials for \emph{asebia} non-belief in the gods in which the \emph{polis} believes is called \emph{adikein} \ldots all private cults are to be forbidden under the law of \emph{asebia} \ldots As long as the ancient \emph{polis} endured \emph{asebia}, the failure to worship the city gods, was a breach of its order. Taking part in the national cult was \emph{eusebeia}, refusal to do so \emph{asebeia}. To the end of antiquity, then, an important part of \emph{eusebeia} or \emph{asebeia} was participation in the national cult or refusal to do this.'' Also, for the pagans, ``As \emph{eusebeia} developed from reverence for the gods and the orders protected by them to worship of the gods, so \emph{asebeia} developed from a lack of reverence for the gods  to neglect of the cultus.'' For the Jews, irreverence could be ``With \emph{asebes} there is never any more precise indication of obj. This does not mean that only God is the direct object of \emph{aseb-}. A first pt. to notice is that \emph{aseb-} never denotes a mere attitude but always action, conduct. Hence \emph{asebeia} can often be put in the plur. as issue of specific acts. If the obj. is indicated only with the verb \emph{asebeo}, this is the more surprising in that the group is not used only for cultic or particularly religious acts. A false witness speaks \emph{asebeia} \ldots \emph{asebes} means one who is guilty in the judgment \ldots \emph{asebeo} means transgression of judicial directions.'' \emph{Irreverent} defines as ``lacking the expected respect for official, important, or holy things.''\cdfoot{irreverent}{2023-03-08} Biblically \emph{irreverent} means conduct against God that is not worthy to call worship, because their hands do not conduct what their tongues confess (Jas 1:22--25). When failing to do so, it is applicable to fall away from the Grace of God into the mosaic jurisdiction if they do not turn away from it.
Found in 1~Tim 1:9.
