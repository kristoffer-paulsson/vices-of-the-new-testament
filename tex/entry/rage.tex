\item[Rage,]
\entlbl{rage} 

\grc{ὀργή}
\index[grc]{οργη@\grc{ὀργή}} 
(\textit{orgē}):
\newglossaryentry{orgē}
{
    name=\grc{ὀργή},
    description={i. \entrefgls{rage} ii. \entrefgls{revenge}},
    sort=οργη@\grc{ὀργή}
}
According to Kleinknecht, ``[A.I] \emph{orgē}, post-Homeric, first found in Hes. Op., 304, then common in poetry and prose, is related in stem to \emph{orgaō}/\emph{orgas}, and thus means the `lavish swelling of sap and vigor,' `thrusting and upsurging' in nature, originally gener. [1.] a. the `impulsive nature' of man or beast, esp. the impulsive state of the human disposition, which in contrast to more inward and quiet \emph{ēthos} \ldots breaks forth actively in relation to what it is without. When \emph{orgē} is used of the mind and nature of man, animal and other comparisons point expressly to the natural side of the concept \ldots The female type which god [lower cased] created the fox \ldots another is best likened to the changeable sea in its nature \ldots The character of men are distinct \ldots who sets \emph{orgē}, man's natural disposition, character and bent, in the category as his \ldots In the general and broader sense of individual nature or disposition \emph{orgē} is esp. important in Attic tragedy where it became a tragic element. In \emph{orgē} there is actualized the true or false insight of man which impels him to decisive deeds \ldots Not blind anger, but demonic excess [Greek] of will in the nature of the tragic person, goes hand in hand with \emph{anankē}, necessity and fate. As compared with older use, \emph{orgē} in tragedy has already become more restricted and specialized. \emph{orgē} now has more of the sense of a specific reaction of the human soul. It takes on the sense [2.] b. of anger as the most striking manifestation of powerful inner passion \emph{thymos}. The two terms can now supplement one other \ldots and yet on the other hand  \emph{orgē}, in distinction from \emph{thymos} [\entref{outburst}], is essentially and intentionally oriented to its content, namely, revenge or punishment \ldots In virtue of this \emph{orgē} itself acquired the meaning [3.] c. `punishment'.''\bkfoot{\grc{ὀργή}}{5:383--4}{\tdntKleinknecht{}} 
Totally, rage can be impulsive on the outside like an \emph{outburst}, however, on the inside, being long gone anger over unsettled issues and from the inside being intentional. \emph{Payback} defines as ``an action that punishes someone for something bad that the person did to you; revenge.''\cdfoot{payback}{2023-03-17} Also, \emph{revenge} as ``harm that you do to someone as a punishment for harm that the person has done to you,''\cdfoot{revenge}{2023-03-17} finally, \emph{rage} as ``extreme or violent anger, or a period of feeling such anger.''\cdfoot{rage}{2023-03-17}
Found in Eph 4:31; Col 3:8.
