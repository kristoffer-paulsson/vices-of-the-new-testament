\item[Haughty disdainer,]
\entlbl{haughty disdainer}

\grc{ὑπερήφανος}
\index[grc]{υπερηφανος@\grc{ὑπερήφανος}}
(\textit{hyperēphanos}):
According to Bertram (TDNT 8:525--8) the term denotes ``[A] The meaning `outstanding,' `distinguished' \ldots But both adj. and the derived noun \emph{hyperephania} are used in the main censoriously of pride, arrogance, boasting. \emph{hyperephanos} is between \emph{hybristes} [\entref{hubris}] and \emph{alazon} [\entref{wannabe}], \ldots Unlike \emph{hybristes}, who acts violently in spite of divine and human law \ldots and the \emph{alazon} \ldots, the empty boaster who deceives himself and others by making the most of his advantages, abilities and achievements, the \emph{hyperephanos} is the one who with pride, arrogance, and foolish presumption brags of his position, power  and wealth and despises others. Both adj. and noun are used of men and supermen \ldots, their acts \ldots, or their attitude to others \ldots the \emph{hyperephanos polites}, who is puffed up and annoying, is contrasted with the honorable and respected citizen. \ldots luxuary, ease and opulence are the soil in which \emph{hyperephania} develops as well as \emph{hybris} \ldots the ref. is to private houses which are not only more ornate (\emph{hyperephanos}) than those of the most but are even more lavishly  (\emph{semnos}) furnished than public buildings. In line with such traditions Hell. ethics is usually against \emph{hyperephania}. \ldots [D] In the NT the noun \emph{hyperephania} occurs only once and the adj. \emph{hyperephanos} five times. The usage is shaped by the OT and the Hellenistic Jewish  tradition with its list of vices. \ldots The group of divinely hated despisers of men, the arrogant and the boastful (or arrogant boasters) is interrelated. According to the exposition of the early Church the arrogant are those who brag of what they have to the have-nots. In the list of vices in 2~Tim 3:2 \emph{alazones} comes before \emph{hyperephanoi}. For the author the two terms denote different form of arrogance. The moral chaos depicted in the list, which is distinguished by a pious exterior, arises out of false teaching and characterizes the last time. The list of vices in Mark 7:20--23, coming after 7:18 f., serves to elucidate further the saying of Jesus about purity (7:15) for the Hellenistic Christian community. The substantive \emph{hyperephania} in 7:22 comes between \emph{blasphemia} [\entref{blasphemy}] `blasphemy' (of God) and \emph{aphrosyne} [\entref{foolishness}] the ungodly attitude of fools \ldots In the first instance \emph{hyperephania} to is against God and stands in contrast to the humility which is proper in relation to God and which is full surrender to Him. It is pride in one's own being and work which already in the OT tradition \ldots denotes resistance to God and the haughty disdain with which others are treated.'' \emph{Haughty} defines as ``unfriendly and seeming to consider yourself better than other people''\cdfoot{haughty}{2023-03-27} and \emph{disdain} is defined ``dislike of someone or something that you feel does not deserve your interest or respect.''\cdfoot{disdain}{2023-03-27}
Found in Rom 1:30; 2~Tim 3:2.
