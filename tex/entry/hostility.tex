\item[Hostility,]
\entlbl{hostility}

\grc{ἔχθρα}
\index[grc]{εχθρα@\grc{ἔχθρα}} 
(\textit{echthra}):
\newglossaryentry{echthra}
{
    name=\grc{ἔχθρα},
    description={i. \entrefgls{hostility} ii. \entrefgls{enmity}},
    sort=εχθρα@\grc{ἔχθρα}
}
According to Foerster, the term denotes ```Hatred,' `hostility,' as an inner disposition, as objective opposition and as actual conflict between nations, groups and individuals. \ldots In the NT \emph{echthra} (in the plur. instances of \emph{echthra}) as enmity between men is one of the works of the flesh along with \emph{eris} [\entref{quarrel}], \emph{zēlos} [\entref{fanaticism}], \emph{thymos} [\entref{outburst}]. \ldots The Law means enmity for man, i.e., enmity between men and enmity against God (not God's enmity against us as in Gal 3:10, but ours against God as in Rom 8:7).''\bkfoot{\grc{ἔχθρα}}{2:815}{\tdntFoerster{}} 
\emph{Hostility} defines as ``an occasion when someone is unfriendly or shows that they do not like something,''\cdfoot{hostility}{2023-03-17} also \emph{enmity} as ``a feeling of hate''\cdfoot{enmity}{2023-04-17}
Found in Gal 5:20.
