\item[Abominable,]
\entlbl{abominable}

\grc{βδελύσσομαι}
\index[grc]{βδελυσσομαι@\grc{βδελύσσομαι}}
(\textit{bdelyssomai}):
\newglossaryentry{bdelyssomai}
{
    name=\grc{βδελύσσομαι},
    description={\entrefgls{abominable}},
    sort=βδελυσσομαι@\grc{βδελύσσομαι}
}
Liddell denotes ``\emph{feel a loathing for food} \ldots \emph{to be sick}, \ldots \emph{feel a loathing at}, \ldots \emph{cause to stink}, \emph{make loathsome} or \emph{abominable}, \ldots \emph{to be loathsome}, \ldots \emph{the abominable}.'' 
Then, Thayer connotes ``\emph{bdelyssō}: (\emph{bdeō} quietly to break wind, to stink);  1. \emph{to render foul}, \emph{to cause to be abhorred} \ldots to defile, pollute \ldots \emph{abominable}, \ldots 2. \emph{bdelyssomai}; deponent middle (1 aorist \emph{ebdelyxamēn} often in the Septuagint \ldots properly, \emph{to turn oneself away from on account of the stench}; metaphorically, \emph{to abhor}, \emph{detest}.'' 
And Gingrich agrees on \textit{abominable}. 
It should be noted that NA28 uses \emph{bdelyssomai} and BNM uses \grc{βδελύσσω} \emph{bdelyssō}. Thayer finds this word to be \emph{deponent middle} which indicates a) that the verb should be understood as active, as well as b) that it is middle, the action is mutual. According to BNM, the word is \emph{participle middle}\footnote{Rev 21:9; word ``7. \grc{ἐβδελυγμένοις} \grc{βδελύσσω} - vpxmdmp (verb participle perfect middle dative masculine plural),'' (BNM) \emph{BibleWorks 10.0.8.755} BibleWorks, 2017.}, where the participle puts the verb in the adjectival and middle as mutual to one another. In both cases, it is the men that consider the saints of God to be \emph{abominable}, while the saint apprehends their opponents as doing what is \emph{abominable}. In both ways, the opponents and the saints have a mutual understanding of one another. \textit{Abominable} defines as ``very bad or unpleasant.''\cdfoot{abominable}{2023-03-13}
Found in Rev 21:8.

