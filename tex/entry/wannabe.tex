\item[Wannabe,]
\entlbl{wannabe} 

\grc{ἀλαζών}
\index[grc]{αλαζων@\grc{ἀλαζών}} 
(\textit{alazōn}):
According to Delling (TDNT 1:226), it denotes the following thing, ``makes more of himself,'' ``ascribing to himself either more and better things than he has, or even what he does not possess at all,'' ``promises what he can not perform.'' Often in the Greek world, the orator, philosopher, poet, magician, doctor, cook, and officer are called this. Liddell says emph{vagabond}, \emph{false pretender}, \emph{impostor}, \emph{quack}, \emph{swaggering}. Thayer and Danker add \emph{braggart} and \emph{boaster} with L. Most terms don't confluence semantically with what D. describes, or only in part. \emph{Wannabe} defines as ``a person who wants to be like someone else, esp. someone famous, or who wants to be thought of as famous,''\cdfoot{wannabe}{2023-03-08} but also defines \emph{pretentious} as ``trying to appear or sound more important or clever than you are, especially in matters of art and literature,'' and ``trying to give the appearance of great importance, esp. in a way that is obvious,''\cdfoot{pretentious}{2023-03-08} also \emph{pretender} as ``a person who states they have a right to the high position that someone else has, although other people disagree with this.''\cdfoot{pretender}{2023-03-08}
Found in Rom 1:30; 2~Tim 3:2.
