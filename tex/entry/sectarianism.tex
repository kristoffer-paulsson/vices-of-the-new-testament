\item[Sectarianism,]
\entlbl{sectarianism} 

\grc{αἵρεσις}
\index[grc]{αιρεσις@\grc{αἵρεσις}} 
(\textit{hairesis}):
Danker describes it as a choice of association based on shared principles or beliefs, ordinarily of a subgroup with views or beliefs that deviate in certain respects from those of a larger party. Usually, there are religious or schismatic sects that rise in heresy and cause divisions in the Body of Christ contrary to the gospel. Schlier (TDNT 1:181) adds that \grc{αἵρεσις} denotes names for both \emph{teachings} and \emph{schools}. For a group to exist the required aspects are usually a comprehensive gathering in the group, a self-chosen authority of a teacher, and an authoritarian doctrine that is well-disputable compared to the Word of God. Thayer inclines that the verb form \grc{αἱρέω} denotes \emph{the act of taking capture}, \emph{the storming of a city}. From that, it's conclusive that unnecessary divisions in a group are about making a coup based on disputably misunderstood teachings. \emph{Sectarianism} defines as ``very strong support for the religious or political group that you are a member of, which can cause problems between different groups.''\cdfoot{sectarianism}{2023-03-05}
Found in Gal 5:20.
