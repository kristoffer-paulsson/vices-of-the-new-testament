\item[Unfaithful (to God),]
\entlbl{unfaithful} 

\grc{ἄπιστος}
\index[grc]{απιστος@\grc{ἄπιστος}} 
(\textit{apistos}):
The word does exist in Greek both in the secular and Christian sense. According to Bultmann, the original \grc{πιστος} in the secular sense denotes ``\emph{Pistos}. In lit. this first means  a. `trusting' \ldots Inasmuch as trust may be a duty, \emph{pistos} can come to have the nuance `obedient' b. \emph{pistos} in the sense `trustworthy' is a word first used in the sphere of sacral law \ldots The expression \ldots denotes the conclusion of a treaty. \emph{ta pista} is the reliability of those bound by the treaty \ldots `fidelity' \ldots Similarly \emph{pistos} (`trustworthy', `faithful') is used of those who stands in a contractual relation.''\bkfoot{\grc{πιστος}}{6:175}{\tdntBultmann{}}
However, contrary as in the said sense B. describes ``\emph{apistos} \ldots `faithless' \ldots `without trust or confidence' \ldots `unworthy of credence' \ldots \emph{apisteō} \ldots `to refuse to believe', \ldots \emph{apistia} `unfaithfulness' \ldots `unbelief'.''\bkfoot{\grc{ἄπιστος}}{6:204--5}{\tdntBultmann{}}
\emph{Faithless} defines as ``not loyal and not able to be trusted,''\cdfoot{faithless}{2023-03-11} and \emph{unfaithful} as ``not loyal or able to be trusted,''\cdfoot{unfaithful}{2023-03-11} finally \emph{unbelief} as ``the fact of not having religious belief.''\cdfoot{unbelief}{2023-03-11}
Found in Rev 21:8.
