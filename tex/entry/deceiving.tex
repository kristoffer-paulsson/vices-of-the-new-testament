\item[Deceiving,]
\entlbl{deceiving}

\grc{ψευδής}
\index[grc]{ψευδης@\grc{ψευδής}}
(\textit{pseudēs}):
According to Conzelmann, the term denotes ``[A.1] c. the adj. \emph{pseudes} means trans. `deceiving,' of persons, \ldots dreams \ldots oracles \ldots, and in the pass. `deceived,' \ldots intr. `untrue,' `false,' `fabricated.' \ldots The adv. also means `falsely,' \ldots `mistakenly'.''\bkfoot{\grc{ψευδής}}{9:595}{\tdntConzelmann{}} Totally in the mentioned context, it is an adj. thereby \emph{deceiving}, which denotes an important rule, not to be deceiving to other people. \emph{Deceiving} defines as ``to persuade someone that something false is the truth; trick or fool.''\cdfoot{deceiving}{2023-03-29}
Found in Rev 21:8.
