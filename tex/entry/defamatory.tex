\item[Defamatory,]
\entlbl{defamatory}

\grc{λοίδορος}
\index[grc]{λοιδορος@\grc{λοίδορος}}
(\textit{loidoros}):
\newglossaryentry{loidoros}
{
    name=\grc{λοίδορος},
    description={\entrefgls{defamatory}},
    sort=λοιδορος@\grc{λοίδορος}
}
According to Hanse (TDNT 4:293--4), ```to reproach,' `insult,' `revile,' even `blaspheme,' though it is not a religious term. In public life in Greece insult and calumny played a considerable part, whether among the heroes in Hom., in political life in the democracies, in comedy, or in the great orators. Not to be susceptible was part of the art of living, though in fact a great deal of objective harm was done by this love of denigration. \ldots The idea that abuse which injures the reputation is a preliminary form of murder reminds us of the exposition of the 6th commandment in the Sermon on the Mount.''\bkfoot{\grc{λοίδορος}}{4:293}{\tdntHanse{}} Thereby it is a form of persecution that the reviler practices, and which believers are said not to hang around this type of persons. \emph{Defamatory} defines as ``damaging the reputation of a person or group by saying or writing bad things about them that are not true.''\cdfoot{defamatory}{2023-03-06} Simply the one who persecutes by defamatory means.
Found in 1~Cor 5:11, 6:10.
