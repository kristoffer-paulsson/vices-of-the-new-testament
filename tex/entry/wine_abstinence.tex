\item[Wine abstinence,]
\entlbl{wine abstinence} 

\grc{οἶνος πολύς προσέχω}
\index[grc]{οινος πολυς προσεχω@\grc{οἶνος πολύς προσέχω}} 
(\textit{oinos polys prosechō}):
\newglossaryentry{oinos polys prosechō}
{
    name=\grc{οἶνος πολύς προσέχω},
    description={\entrefgls{wine abstinence}},
    sort=οινος πολυς προσεχω@\grc{οἶνος πολύς προσέχω}
}
All Liddell, Thayer, and Gingrich, including Danker, agree on \grc{πολύς} meaning \emph{great}. Then T. and G. basically denote \grc{προσέχω} meaning \emph{addicted to}. However, it is strange to prohibit deacons from having heavy addiction if \entref{drunkenness} is a sin in general and lower to reach when he belongs higher up than the everyday members of the congregation. Both L. T. and G. agree with \emph{to turn one's mind to} which is very fortunate from a doctrinal point of view. Instead of focusing on great addiction, a different nuance, based on abstinence, can be addressed. Supposedly it can translate as \emph{greatly turning one's mind to wine}, which would indicate abstinence from frequently drinking wine in smaller portions but without completely controlling it. \emph{Abstinence} defines as ``the act of not doing something, esp. something that gives you pleasure.''\cdfoot{abstinence}{2023-03-23}
Found in 1~Tim 3:8.
