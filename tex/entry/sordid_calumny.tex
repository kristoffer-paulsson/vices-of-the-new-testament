\item[Sordid calumny,]
\entlbl{sordid calumny} 

\grc{αἰσχρολογία}
\index[grc]{αισχρολογια@\grc{αἰσχρολογία}} 
(\textit{aischrologia}):
Stems from two words: \grc{αἰσχρός} \emph{sordid}, and \grc{λόγος} which according to Debrunner (TDNT 4:73) aims for \emph{assembling}, \emph{reckoning}, \emph{account}, \emph{review}, and \emph{narrative}. Delling (TDNT 4:4) describes the contrast of \grc{καταλαλιά} which is: ``evil report'' and ``calumny'' as ``malicious or unthinking gossip.'' Thereby sordid accusations or evil thought-through slander cunningly planted. Sordid \emph{calumny} defines as ``(the act of making) a statement about someone that is not true and is intended to damage the reputation of that person.''\cdfoot{calumny}{2023-03-05} Also, see \entref{sordid}. Simply calumny that portrays someone as sordid or shameful.
Found in Col 3:8.
