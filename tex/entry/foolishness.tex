\item[Foolishness,]
\entlbl{foolishness}

\grc{ἀφροσύνη}
\index[grc]{αφροσυνη@\grc{ἀφροσύνη}}
(\textit{aphrosynē}):
According to Bertram (TDNT 9:22-35), firstly \grc{ἀφρων} means \emph{without understanding}, \emph{befooled}, \emph{mad}, \emph{out of one's mind}, \emph{lack of understanding}, and secondly \grc{ἀφροσύνη} means \emph{to be irrational}, \emph{youthful folly}. There are two main understandings, the religious and the worldly. In the religious understanding, a person who commits acts or sins in a way that may cut him off from God is foolish. On the other hand, worldly foolishness is very grave and which acts leads to lynching. \emph{Foolishness} defines as ``the quality of being unwise, stupid, or not showing good judgment.''\cdfoot{foolishness}{2023-03-12} Also, \emph{madness} as ``stupid or dangerous behavior.''\cdfoot{madness}{2023-03-12}
Found in Mark 7:22.
