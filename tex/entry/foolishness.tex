\item[Foolishness,]
\entlbl{foolishness}

\grc{ἀφροσύνη}
\index[grc]{αφροσυνη@\grc{ἀφροσύνη}}
(\textit{aphrosynē}):
\newglossaryentry{aphrosynē}
{
    name=\grc{ἀφροσύνη},
    description={\entrefgls{foolishness}},
    sort=αφροσυνη@\grc{ἀφροσύνη}
}
According to Bertram, firstly \grc{ἀφρων} means ``[A.1] `without understanding', \ldots [A.2] `befooled', \ldots `mad', `out of one's mind',  \ldots `lack of understanding',''\bkfoot{\grc{ἀφρων}}{9:220-1}{\tdntBertram{}}
and secondly \grc{ἀφροσύνη} means ``[A.2] `youthful folly,' \ldots [A.3.b] In man, however, \emph{aphrosynē} is either the animal by nature or sick as epilepsy or mania. \ldots Thus the honest man is rational and the liar foolish.''\bkfoot{\grc{ἀφροσύνη}}{9:221-2}{\tdntBertram{}}
There are two main understandings, the religious and the worldly. In the religious understanding, a person who commits acts or sins in a way that may cut him off from God is foolish. On the other hand, worldly foolishness is very grave and which acts leads to lynching. \emph{Foolishness} defines as ``the quality of being unwise, stupid, or not showing good judgment.''\cdfoot{foolishness}{2023-03-12} Also, \emph{madness} as ``stupid or dangerous behavior.''\cdfoot{madness}{2023-03-12}
Found in Mark 7:22.
