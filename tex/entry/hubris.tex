\item[Hubris,]
\entlbl{hubris}

\grc{ὑβριστής}
\index[grc]{υβριστης@\grc{ὑβριστής}}
(\textit{hybristēs}):
\newglossaryentry{hybristēs}
{
    name=\grc{ὑβριστής},
    description={i. \entrefgls{hubris} ii. \entrefgls{arrogance} iii. \entrefgls{contempt} iv. \entrefgls{forceful interference} v. \entrefgls{ill-will} vi. \entrefgls{injure} vii. \entrefgls{insolent} viii. \entrefgls{insult} ix. \entrefgls{rape} x. \entrefgls{sexual violation} xi. \entrefgls{vainglorious arrogance} xii. \entrefgls{violence of rich to poor}},
    sort=υβριστης@\grc{ὑβριστής}
}
Bertram denotes ``[A] \emph{hybris} is etym. obscure. The second syllable originally connected with \emph{briaros} `weighty,' \emph{brithō} `heavily laden.' Popular etym., as already in Hom., derives it from \emph{hyper} along the lines of `beyond measure.' This is linguistically impossible but important historically. With both noun and verb the range of meaning is very large. The noun means originally an act which invades the sphere of another to his hurt, a `trespass,' a `transgression' of the true norm in violation of divine and human right. Arrogance of disposition is often implied, \ldots The ref. is to a wicked act, also insult, scorn, contempt, often accompanied by violence, rape, and mistreatment of all kinds. \ldots The verb \emph{hybrizō}, which is primarily trans. in formation, has the same range of meaning. \ldots it denotes intr. arrogant conduct and trans. `to harm,' `damage,' `injure,' \ldots the injurious treatment of others even to rough handling. From the class. age it is also common in the pass. \emph{hybristēs}, derived from the verb, denotes a man who, sinfully overestimating his own powers and exaggerating his own claims, is insolent in word and deed in relation to gods and men. \ldots [A.4] In legal rhetoric one finds hubris technically from Aristoph. In trials the main issue is the violence of the rich against the poor, Lys., \ldots One finds violation of personal rights and forceful interference in the personal or domestic sphere, \ldots Aeschin. Tim., 15 tbe \emph{hybreōs graphē} contains the law of injuries and violations which exerted considerable influence and shaped the similar law in Alexandria. \ldots [A.6] With the mythical or philosophical usage of earlier times  (which continues to exert an influence) and with the legal use, one also finds new application in Aristot. In him, as earlier \ldots, \emph{hybris} can be sexual violation, \ldots As an expression of content \emph{hybris} means `maltreatment' with \emph{kataphronēsis} `scorn' and \emph{epēreasmos} `ill-will,' \ldots is also used act. for `arrogance' as a form of \emph{adikia} [\entref{unrighteousness}] \ldots with \emph{asebeia} [\entref{godless}] `offense against gods and men' and \emph{pleonexia} [\entref{unscrupulous encroaching}] \ldots 'greed,' \ldots Hubris in action is associated with pleasure, \ldots The high-minded man is not to be confused with the arrogant, \ldots To act arrogantly is very wrong, but hubris cannot be punished, for a presumptuous disposition is a gen. human complaint to which some  (the rich and young) are more prone and others less. \ldots [A.7] The later period brings no essential changes \ldots as already in Demosth. \ldots \emph{aselgēs} [\entref{brutal}] and \emph{hybrizōn} are synon. \ldots characterizes the \emph{hybristēs} as by nature prone to (mocking) laughter \ldots speaks of hubris in intercourse \ldots of hubris against the law \ldots Thus hubris passed into common usage in many senses, some of them quite weak. It retains a certain emotional force as a poetic term. But in an age when more and more the problem of ethics and anthropology were being considered with the tools of rational thought this word which originally owed its content to myth could not become a tt. in philosophy. Hence hubris never became a key concept in Gk. thought. \ldots [B.3] Since hubris is so broad and can denote disposition, attitude and conduct, sinful turning from or provocation of God, secularism, as well as vainglorious arrogance, encroachments and tyranny against one's fellows, it is very hard to fix the limits of signification whether over against synon. or related Gk. words or with ref. to the equivalent Hbr. roots. In fact many Hbr. roots stand close in sense to hubris or are in context an expression of it. Thus one may ref. to \ldots `to be great, lofty, exalted,' but also `to be boastful, proud, arrogant'.''\bkfoot{\grc{ὑβριστής}}{8:295--301}{\tdntBertram{}}
 \emph{Hubris} defines as ``a way of talking or behaving that is too proud,'' and as ``an extreme and unreasonable feeling of pride and confidence in yourself.''\cdfoot{hubris}{2023-03-27}
Found in Rom 1:30; 1~Tim 1:13.
