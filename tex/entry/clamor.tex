\item[Clamor,]
\entlbl{clamor}

\grc{κραυγή}
\index[grc]{κραυγη@\grc{κραυγή}}
(\textit{kraugē}):
According to Grundmann, \grc{κραυγή} denotes ``The meaning is a. `to croak or cry with a loud and raucous voice' \ldots (groan deeply) \ldots (to be dissatisfied) \ldots It is a war-cry \ldots (of the capture of Jericho). \ldots A second sense b. `to demand with cries'.''\bkfoot{\grc{κραυγή}}{3:898}{\tdntGrundmann{}} Then Liddell, Thayer, and Gingrich almost agree in \emph{clamor} or \emph{outcry}. \emph{Outcry} defines as ``a strong expression of anger and disapproval about something, made by a group of people or by the public,''\cdfoot{outcry}{2023-03-20} and \emph{clamor} as ``a loud complaint about something or a demand for something.''\cdfoot{clamor}{2023-03-20}
Found in Eph 4:31.
