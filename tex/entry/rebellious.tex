\item[Rebellious,]
\entlbl{rebellious} 

\grc{ἀνυπότακτος}
\index[grc]{ανυποτακτος@\grc{ἀνυπότακτος}} 
(\textit{anypotaktos}):
\newglossaryentry{anypotaktos}
{
    name=\grc{ἀνυπότακτος},
    description={\entrefgls{rebellious}},
    sort=ανυποτακτος@\grc{ἀνυπότακτος}
}
According to Delling, it denotes ```not subject,' \ldots `free,' \ldots `not subjected to specific (here geographical) ideas.' \ldots `not subjecting' oneself to sound teaching or its proponents (Titus 1:9--10). \ldots`not submissive' either in practice or principle. Here too, as in current usage, it has the sense of `rebellious' or `refractory' against God's will (1~Tim 1:9).''\bkfoot{\grc{ἀνυπότακτος}}{8:47}{\tdntDelling{}}
\emph{Refractory} is defined as ``not affected by a treatment, change, or process.''\cdfoot{refractory}{2023-03-08} Also, \emph{rebellious} is defined as ``If a group of people are rebellious, they oppose the ideas of the people in authority and plan to change the system, often using force.''\cdfoot{rebellious}{2023-03-08} When people rebel, it is applicable to fall away from the Grace of God into the mosaic jurisdiction if they do not submit.
Found in 1~Tim 1:9.
