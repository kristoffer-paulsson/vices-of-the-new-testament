\item[Sinner,]
\entlbl{sinner} 

\grc{ἁμαρτωλός}
\index[grc]{αμαρτωλος@\grc{ἁμαρτωλός}} 
(\textit{hamartōlos}):
According to Rengstorf (TDNT 1:317), \grc{ἁμαρτωλός} denotes the core connotation as \grc{ἁμαρτία} which is ``not hitting,'' ``missing,'' ``the man who misses something,'' and \grc{ἁμαρτωλός} specifically ``sinner,'' ``when he misses something.'' Except that there is no satisfying or distinguishing utility to explain such properly. Here we must explicitly rely on the context of 1~Tim 1:9. The context says that the mosaic law is for those sinners but can impossibly refer to sinners of the mosaic law itself. Therefore only the Law of Christ (1~Cor 9:21, Gal 6:2) in the NT can be applicable (Matt 22:35--40, John 13:24). When failing to do so, it is applicable to fall away from the Grace of God into the mosaic jurisdiction if they do not turn away from it.
Found in 1~Tim 1:9.
