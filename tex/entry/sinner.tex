\item[Sinner,]
\entlbl{sinner} 

\grc{ἁμαρτωλός}
\index[grc]{αμαρτωλος@\grc{ἁμαρτωλός}} 
(\textit{hamartōlos}):
\newglossaryentry{hamartōlos}
{
    name=\grc{ἁμαρτωλός},
    description={\entrefgls{sinner}},
    sort=αμαρτωλος@\grc{ἁμαρτωλός}
}
According to Rengstorf, \grc{ἁμαρτωλός} denotes the core connotation as \grc{ἁμαρτία} which is ```not hitting,' or `missing.' \emph{hamartōlos} is thus `the man who misses something.' That is to say, he is \emph{hamartōlos} `when he misses something'.''\bkfoot{\grc{ἁμαρτωλός}}{1:317}{\tdntRengstorf{}} 
Except that there is no satisfying or distinguishing utility to explain such properly. Here we must explicitly rely on the context of 1~Tim 1:9. The context says that the mosaic law is for those sinners but can impossibly refer to sinners of the mosaic law itself. Therefore only the Law of Christ (1~Cor 9:21, Gal 6:2) in the NT can be applicable (Matt 22:35--40, John 13:24).
Found in 1~Tim 1:9.
