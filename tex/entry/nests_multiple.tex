\item[Nests (multiple),]
\entlbl{nests} 

\grc{κοίταις}
\index[grc]{κοιταις@\grc{κοίταις}} 
(\textit{koitais}, plural):
Liddell mentions a long array of cognates regarding to \grc{κοίτ-} that needs to be examined. Sadly the TDNT has no given articles regarding \grc{κοίτη}, therefore a deep specific study is needed, \grc{κοίτη} ``\emph{bedstead},\ldots \emph esp. \emph{marriage-bed}, \ldots also \ldots, of a cave, \ldots on a \emph{sick-bed}, \ldots of one dead, \ldots \emph{lair} of a wild beast, \emph{nest} of a bird, etc., \ldots of the spider, \ldots of the fish \ldots \emph{quarters}, \ldots \emph{pen}, \emph{fold} for cattle, \ldots \emph{act of going to bed}, \ldots to entertain `at bed and board', \ldots for \emph{going to bed}, \ldots \emph{to lie still} in death, \ldots of \emph{sexual connexion}, \ldots to become pregnant by a man.'' What can be concluded is that the term relates to what happens at night or at the sleeping quarters, everything concluded. The cognates follows: \grc{κοῖτος} ``\emph{resting-place}, \emph{bed}, \ldots of birds,\ldots \emph{stall}, \emph{fold}, \ldots \emph{sleep},\ldots \emph{bed-time}, \ldots \emph{lying abed} till dawn, \ldots sleep under arms, \ldots go to \emph{bed}.'' \grc{κοιτωνικός} ``\emph{for a bedroom}, \ldots \emph{bed-cover},'' \grc{κοιτωνιάρχης} \emph{chamberlain}, \grc{κοιτωνίτης} \emph{chamberlain}, \grc{κοιτωνοφύλαξ} ``\emph{guardian of the bed-chamber}'' \grc{κοιτάζω} ``\emph{put to bed}, \ldots esp. of cattle, \emph{fold}, \ldots \emph{cause to rest}, \ldots \emph{go to bed}, \emph{sleep}, \ldots \emph{have a lair}, of a lion, \ldots \emph{nest}, of birds.'' \grc{κοιτάριος} ``\emph{for beds},'' \grc{κοιτασία} \emph{cohabitation}.
Men that have \grc{κοίτη} are not men of sinful conduct but instead have a home in place, just like a male bird has to build a nest for his darling. Then in the nest, the birds cohabit and raise the chicks -- which after some time -- leave the nest to try their wings. It is all about expected human and mammal conduct of reproduction. A behavior that is most natural for everything that lives that has to reproduce.
When people of the Stoneage epoch were alive, they sought protection from the dangers of the night and started to look for a safe place to live and protect. Surrounding their protective resting place, they furnished what later became their home and raised their children from there.
In the Bible, we read about David and Bathsheba (2~Sam 11). When David sent for Uriah, and he later went asleep at the entrance of the palace, the LXX uses (v9) the word \grc{κοιμάομαι} \emph{fall asleep}. The second time Uriah went back (v13) to sleep, the used word is \grc{κοίτη} \emph{bedstead} because he now had a dedicated resting place where he slept at the palace.
In Gen 35:22, we read about Reuben sleeping with his Father Israel's concubine Bilhah, here the LXX uses the word \grc{κοιμάομαι}, but later in Gen 49:4, when Israel is removing Reuben's firstborn rights, he accuses him of climbing Israel's \grc{κοίτη} \emph{sexual connection}. It is noteworthy that Reuben also climbed and defiled his Father's \grc{στρωμνή} \emph{mattress}, thereby we can separate the act of sleeping with someone as in no relationship and the sexual connection that is similar to marriage. Also, we see the difference between the bed and the relationship playing out in the bed!
In Rom 9:10 it says ``not only but except even Rebecca out of one \emph{cohabitation} conceived, with our father Isaac (NA28; my translation).'' What I want to push is that a \grc{κοίτη} between a man and a woman is always about a sexual connection with cohabitation included. What we are looking for is the conduct of how humans go about living with each other, and that it always starts with a regular bed that they share in common, and mutually agreed cohabitation happens, then there is the wedding ceremony which establishes the union before witnesses and society as official.
 Technically \grc{κοίτη} as \emph{nest} is in itself human conduct. The purpose of this article is to conceive the understanding of nests or \emph{chambering} - in this case - it's the plural version that is sinful, that is, if someone has more than one sexual connection that includes cohabitation in terms of sleeping over more than one night at a time. Polygamy is thereby prohibited, whether formally married or NOT!
\emph{Nest} defines as ``to build a nest, or live in a nest,'' and ``a structure or other place where creatures, esp. birds, give birth or leave their eggs to develop.''\cdfoot{nest}{2023-03-10} Also, \emph{polygamy} defines as ``the custom or condition of being married to more than one person at the same time.''\cdfoot{polygami}{2023-03-10}
Found in Rom 13:13.
