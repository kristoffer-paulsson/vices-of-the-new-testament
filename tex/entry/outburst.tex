\item[Outburst,]
\entlbl{outburst} 

\grc{θυμός}
\index[grc]{θυμος@\grc{θυμός}} 
(\textit{thymos}):
According to Büchsel, said term denotes ``From the sense of `to well up,' `to boil up,' there seems to have developed that of `to smoke,' and then `to cause to go up in smoke,' `to sacrifice.' The basic meaning of \emph{thymos} is thus similar to that of \emph{pneuma}, namely, `that which is moved and which moves,' `vital force.' \ldots \emph{thymos} then takes on the meaning sense of a. desire, impulse, inclination, b. spirit, c. anger, d. sensibility, e. disposition or mind, f. thought, consideration. \ldots [for the writers] For them \emph{thymos} means spirit, anger, rage, agitation.''\bkfoot{\grc{θυμός}}{3:167}{\tdntBuchsel{}} 
\emph{Outburst} defines as ``a sudden forceful expression of emotion, especially anger.''\cdfoot{outburst}{2023-03-16}
Found in Gal 5:20; Eph 4:31; 2~Cor 12:20; Col 3:8.
