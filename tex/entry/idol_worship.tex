\item[Idol worship,]
\entlbl{idol worship} 

\grc{προσκυνέω ὁ εἴδωλον}
\index[grc]{προσκυνεω ο ειδωλον@\grc{προσκυνέω ὁ εἴδωλον}} 
(\textit{proskyneō ho eidōlon}):
According to Büchsel (TDNT 2:375--8), idol and idolatry can be well denoted ``[1.] 1. \emph{eid-olon}, from \emph{eid-}, \emph{vidare}, `to see' \ldots means `picture' or `copy,' whether artificially made, self-produced or simply present. Thus \emph{eidolon} might mean `figure of a man' in the sense of a copy of the man depicted, but not the man himself. It can also be used for images of gods etc. On the other hand, the usual term for the cultic images of the Greeks is \emph{agalma}, while the statues of men are normally called \emph{andrias} and \emph{eikon}. The relevant cultic object as such is not \emph{eidolon}, but the relation to the deity can be formulated in such way that it is  not \emph{eidolon}. It helps us understand the Greek conception and sense of the word that they call reflections in water \ldots and that they also call the shadow \emph{eidolon skias}. \emph{eidolon} can also be used for shades or apparitions, and is par. to \emph{phasma}. Indeed, the inhabitants of the underworld are called \emph{eidola}, though they are no longer the men concerned, but only copies of them. A work of art is called \emph{eidolon} in the sense of an unconscious and immobile copy quite distinct from the living being in question. \ldots \emph{eidolon} can also denote the image awakened by an object in the soul. It is common in Philo in the sense of what is unreal or deceptive. Though it would be too much to equate it with what is without substance, it certainly denotes `copy' as distinct from the true reality. [2.] 2. The LXX uses \emph{eidolon} for many words meaning images of the gods or heathen deities \ldots A first point to notice is that \emph{eidolon} rather than \emph{agalma} is used for images of gods. Even more important is that the term is applied to the gods themselves. Behind the usage there is obviously a polemic against paganism. The presence of the images as the focus of worship is used to emphasize the unreality of heathen belief and the heathen gods. For the Jews idols and heathen deities are identical, and they prove that the heathen have images but no true God. Thus `copy' (as distinct from the reality) is the word for both images and gods. The word `idol' in its current use does not always convey the precise meaning. Too great emphasis is often laid on the idea of an object of false worship rather than on that of something without reality which fools have put in the place of the true God. In its strict sense the idol is not merely an alternative god; it is an unreal god, and therefore false as distinct from true and real. Philo and Josephus are both familiar with this use of \emph{eidolon}. \ldots In pagan Gk. we do not find this usage. The Gks. did not share this view. For either they honor as gods what Jews call \emph{eidolon}, or, even if they no longer do so, they have no comprehensive expression for what the Jews call \emph{eidolon}. The language of the LXX is biblical or Jewish Gk. in this respect. Jewish religion has coined a new expression out of an existing term. \ldots [3.] 3. The NT usage rests on that of the LXX or the Jews. In the NT \emph{eidolon} is used for heathen gods and their images. The word and its derivatives do not occur on the Gospels. \ldots In relation to Paul's use of \emph{eidolon} the question arises how far he regards them as realities. It is evident from 1~Thess 1:9 that they are \underline{no gods} [emphasis added] in comparison with God, and from Gal 4:8 and Rom 1:23 that they are \underline{not divine} [emphasis added] by nature but only products of human sin and folly. But he seems to \underline{see demons behind their worship} [emphasis added] (1 Cor 10:19; cf. 8:5), so that we do not have here a purely intellectual dismissal. He gave full weight to Deut 32:17: \emph{ethusan daimoniois kai ou theo}. In this respect he is wholly Jewish.'' Paul denotes that \entref{idol worship} or \entref{idolatry} is actual demon worship and is forbidden both in OT and NT. \emph{Worship} defines as ``to love, respect, and admire someone or something very much, often without noticing the bad qualities of that person or thing.''\cdfoot{worship}{2023-03-16}
Found in Acts 15:20; Rev 9:20.
