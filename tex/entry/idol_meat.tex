\item[Idol meat,]
\entlbl{idol meat}

\grc{εἰδωλόθυτος}
\index[grc]{ειδωλοθυτος@\grc{εἰδωλόθυτος}}
(\textit{eidōlothytos}):
According to Büchsel (TDNT 2:378--9), it denotes ``the meat which derives from heathen sacrifices, though without the intolerable implication of the sanctity of what is offered to heathen gods, or the divinity of gods. \ldots Paul allow the enjoyment of \emph{eidolothutos} apart from the cultic act itself (1~Cor 10:14--22) and so long as it does not violate the law of love (8:1--13). In this connection he appeals (10:26) to Ps 24:1. He can take this attitude only because faith has overcome Jewish legalism from within. In the apostolic decree of Acts 15:29; 21:25, and in Rev 2:14, 20, we do not have full freedom from legalism. Among the Nicolaitans the desire the eat meat sacrificed to idols is an exception of Libertarianism, i.e., of complete renunciation of any commitment to the will of God, as may be seen from their general licentiousness. The same is probably true of Paul's opponents in Corinth.'' However, it is clear that we are not supposed to eat from an idol sacrifice, but if we eat a meal blindly and give thanks to God, we are free to eat by faith without consequence. See \entref{idol worship}.
Found in Acts 15:20, 21:25.
