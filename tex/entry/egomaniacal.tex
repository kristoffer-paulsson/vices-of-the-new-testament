\item[Egomaniacal,]
\entlbl{egomaniacal}

\grc{τυφόω}
\index[grc]{τυφοω@\grc{τυφόω}}
(\textit{typhoō}):
\newglossaryentry{typhoō}
{
    name=\grc{τυφόω},
    description={\entrefgls{egomaniacal}},
    sort=τυφοω@\grc{τυφόω}
}
Thayer denotes ``properly, \emph{to raise a smoke}, \emph{to wrap in a mist}; used only metaphorically:  1. \emph{to make proud}, \emph{puff up with pride}, \emph{render insolent}; passive, \emph{to be puffed up} with haughtiness or pride.'' Then, Gingrich denotes: \emph{be puffed up}, \emph{be conceited}. 
Also Liddell denotes \grc{τυφόω} ``\emph{delude}, \ldots \emph{to be crazy}, \emph{demented}, \ldots \emph{rendered vain}, \ldots \emph{filled with insane arrogance}.''
Then: examining the cognates found in L. gives a broader understanding of what the concept of \grc{τῦφ(ο/ω)-} is, 
\grc{τύφω} ``\emph{smoke}, \ldots \emph{consume in smoke}, \emph{burn slowly}, \ldots \emph{smolder} \ldots \emph{smoldering}, but not yet broken out \ldots also of the fire of love,''
\grc{τῦφος} ``\emph{delusion}, \ldots \emph{colloquially}, \emph{nonsense}, \emph{humbug}, \emph{affectation}, \ldots \emph{vanity}, \emph{arrogance},''
\grc{τυφογέρων} ``\emph{silly old man}, \emph{dotard},''
\grc{τυφομανία} ``\emph{delirium}, \ldots \emph{mad delusion},''
\grc{τυφοπλαστέω} ``\emph{invent a falsehood}, \ldots \emph{deceiving} himself,'' 
\grc{τυφοπλάστης} ``\emph{inventor of falsehood},'' 
\grc{τυφοποιέω} ``\emph{construct an imaginary world},'' 
\grc{τυφώδης} ``\emph{delirious} \ldots also of the fever.'' 
A lot is mentioned about fever but being puffed up doesn't seem like normal pridefulness instead it seems to be a puffed-up delirious ego of vanity. \emph{Vanity} defines as ``used to describe something that is done with the aim of getting praise, fame, or approval rather than for serious or good reasons,''\cdfoot{vanity}{2023-03-24} and \emph{egocentric} as ``thinking only about yourself and what is good for you,''\cdfoot{egocentric}{2023-03-24} else \emph{conceited} as ``too proud of yourself and your actions and abilities,''\cdfoot{conceited}{2023-03-24} and \emph{egomaniacal} as ``relating to someone who considers themselves to be very important and able to do anything that they want.''\cdfoot{egomaniacal}{2023-03-24}
Found in 1~Tim 3:6; 2~Tim 3:4.
