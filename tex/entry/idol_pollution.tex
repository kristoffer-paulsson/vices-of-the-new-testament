\item[Idol pollution,]
\entlbl{idol pollution}

\grc{ἀλίσγημα ὁ εἴδωλον}
\index[grc]{αλισγημα ειδωλον@\grc{ἀλίσγημα ὁ εἴδωλον}}
(\textit{alisgēma ho eidōlon}):
\newglossaryentry{alisgēma ho eidōlon}
{
    name=\grc{ἀλίσγημα ὁ εἴδωλον},
    description={\entrefgls{idol pollution}},
    sort=αλισγημα ειδωλον@\grc{ἀλίσγημα ὁ εἴδωλον}
}
Liddell calls \grc{ἀλίσγημα} \emph{pollution}, and the cognate \grc{ἀλισγέω} ``pollute,'' while other relating words are completely false cognates. Elsewhere in the bible it is only used twice. Daniel decided in his heart that he didn't want to be polluted by the food and wine given by the king of Babylon (Dan 1:8), further, Malachi orates prophetically that the people puts polluted bread/food on his altar and despise his table (Mal 1:7, 12). Mostly it is about what comes from the altar of idolatry, what is sacrificed to demons, it is food only and grace kicks in here according to 1~Cor 10. Also see  \entref{idolater}, \entref{idolatry}, \entref{idol meat}, \entref{idol worship}, and \entref{demon worship}
Found in Acts 15:20.
