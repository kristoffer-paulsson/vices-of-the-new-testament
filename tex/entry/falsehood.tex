\item[Falsehood,]
\entlbl{falsehood}

\grc{ψεῦδος}
\index[grc]{ψευδος@\grc{ψεῦδος}}
(\textit{pseudos}):
According to Conzelmann, the term denotes ``[A.1] b. The noun \emph{pseudes} means `what is untrue,' `deceit,' `falsehood,' `lying,' `lie,' \ldots whether in obj. or the subj. sense is open as in the case of the verb. This leads on to the use in logic on the one side and in the ethics on the other.''\bkfoot{\grc{ψεῦδος}}{9:595}{\tdntConzelmann{}}
The main focus is on deceiving and being untrue. \emph{Untrue} defines as ``not true; false,''\cdfoot{untrue}{2023-03-29} then \emph{falsehood} as ``a lie or a statement that is not correct.''\cdfoot{falsehood}{2023-03-29}
Found in Rev 21:27, 22:15.
