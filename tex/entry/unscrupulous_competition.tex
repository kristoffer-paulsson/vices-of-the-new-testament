\item[Unscrupulous competition,]
\entlbl{unscrupulous competition} 

\grc{πλεονέκτης}
\index[grc]{πλεονεκτης@\grc{πλεονέκτης}} 
(\textit{pleonektēs}):
First, let's explore the core sentiments of \grc{πλέον ἔχειν}. Delling (TDNT 6:266--7) denotes \emph{having more}, \emph{receiving more}, \emph{wanting more}; after that, it connotes the following: \emph{striving for power}, \emph{to take the greater share}, \emph{to increase one's possessions}, \emph{to seek aggrandizement}, \emph{to take advantage of}, \emph{to seek political gain}; further: \emph{to be in force}, \emph{to be superior} in number and weapons, \emph{to take precedence} in power, \emph{to forge ahead} at the expense of others, \emph{to treat someone arrogantly}.\\Then the meaning of \grc{πλεονέκτης} itself. ``\emph{pleonektēs} is the `robber,' \ldots a. `to be superior' in battle \ldots `to surpass' in numbers, \ldots `to be ahead of someone' in goods, right conduct, \ldots `to excel in something,' of opposite kinds of music, \ldots b. `to receive more' in material distribution, \ldots `to be at an advantage' in dealings with gods, \ldots `to gain' \ldots c. `to gain advantages,' \ldots `to take advantage of someone,' \ldots `to seize the goods of others,' `to seek something by force,' \ldots `to do violence to,' laws.'' Based on distinguishing between \emph{unscrupulous competition} and \entref{unscrupulous encroaching}, the competition is an active choice and hunt for wealth compared to stealing indifferently from the surroundings of others. Therefore insatiability is a common factor but still different in its approach. We let \emph{unscrupulous} be defined as ``behaving in a way that is dishonest or unfair in order to get what you want,''\cdfoot{unscrupulous}{2023-03-24} and \emph{competition} as ``a situation in which someone is trying to win something or be more successful than someone else.''\cdfoot{competition}{2023-03-24}
Found in Eph 5:5; 1~Cor 5:10, 11, 6:10.
