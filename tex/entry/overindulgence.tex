\item[Overindulgence,]
\entlbl{overindulgence} 

\grc{ἡδονή}
\index[grc]{ηδονη@\grc{ἡδονή}} 
(\textit{hēdonē}):
Stählin denotes ``[A.1] The word \emph{hedone} derives from the same root as \emph{hēdus}, `sweet,' `pleasant,' `delightful' \ldots and it shares with this adj. the original sense of what is pleasant to the senses, namely, to the sense of taste. \ldots of superficial `pleasure in rhetoric' (feasting the ears). The narrower basic meaning, which relates to what tastes good, was accompanied for many centuries by a subsidiary semasiological strand according to which \emph{hēdonē} denotes that which causes pleasure to the senses. The specific sense of `pleasant taste' is first found in \ldots Already in its earliest use, however, the term \emph{hēdonē} bears the broader sense of a general `feeling of pleasure' or `enjoyment.' The development from the sensual to the physical and then to the ethical, which we can trace in \emph{hēdonē}, is often to be noted in the evolution of words and their meanings. Already at an early period the rise of a feeling of pleasure is both restricted to sensual perceptions but is linked with enjoyable experiences of all kinds, and esp. with desired communications.''\bkfoot{\grc{ἡδονή}}{2:909--10}{\tdntStahlin{}}
Liddell connotes to earlier mentioned ``\emph{enjoyment}, \emph{pleasure}, \ldots prop. of sensual pleasures, \ldots of \emph{malicious pleasure}, \ldots to give way \emph{to pleasure}, \ldots shall I speak truly or \emph{so as to humor you?} \ldots one feels \emph{pleasure} at the thought that \ldots to be \emph{satisfied with} \ldots to speak so as to please another, \ldots it is a \emph{pleasure} or \emph{delight} to another, \ldots to take \emph{pleasure} in them, \ldots with pleasure, \ldots concrete, \emph{a pleasure}, \ldots sweetmeats, \ldots \emph{desires after pleasure}, \emph{pleasant lusts}.'' According to the Bible, illicit gratification is harmful to your walk with God (Luk 8:14). \emph{Hedonism} defines as ``living and behaving in ways that mean you get as much pleasure out of life as possible, according to the belief that the most important thing in life is to enjoy yourself,''\cdfoot{hedonism}{2023-03-16} also \emph{overindulgence} as ``behavior in which you allow someone to have more of something enjoyable than is good for them.''\cdfoot{overindulgence}{2023-03-24}
Found in Titus 3:3.
