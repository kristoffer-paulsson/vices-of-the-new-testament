\item[Selfish triumph,]
\entlbl{selfish triumph} 

\grc{ἀφιλάγαθος}
\index[grc]{αφιλαγαθος@\grc{ἀφιλάγαθος}} 
(\textit{aphilagathos}):
\newglossaryentry{aphilagathos}
{
    name=\grc{ἀφιλάγαθος},
    description={\entrefgls{selfish triumph}},
    sort=αφιλαγαθος@\grc{ἀφιλάγαθος}
}
First, we describe the opposite \grc{φιλάγαθος}, according to Grundmann it describes ``Aristotle calls \emph{philagathos} the man who, in contrast to the \emph{philautos} [\entref{selfishness}] who is \emph{phaulos}, places his ego under the good.''
The said term is denoted according to G. ``This word is distinguished from \emph{philautos} (the opposite of  \emph{philagathos}),  the term which introduces the description, in the sense that it says of \emph{philautoi} that as men who know only themselves they seem to have no knowledge of live or piety. It is part of the NT description of the last time that in it lovelessness celebrates its triumph.''\bkfoot{\grc{ἀφιλάγαθος}}{1:18}{\tdntGrundmann{}}
Usually translated as \emph{no love of good}. However, it could be setting oneself above the love of excellence, which is ego over what is good. People look away from the common good for their selfish desire. \emph{Selfish} defines as ``caring only about what you want or need without any thought for the needs or wishes of other people.''\cdfoot{selfish}{2023-03-12} And \emph{triumph} as ``to have a very great success or victory.''\cdfoot{triumph}{2023-03-12}
Found in 2~Tim 3:3.
