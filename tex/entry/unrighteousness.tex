\item[Unrighteousness,]
\entlbl{unrighteousness} 

\grc{ἀδικία}
\index[grc]{αδικια@\grc{ἀδικία}} 
(\textit{adikia}):
The word stems from \grc{ἄδικος}, described in TDNT (1:149--57) by Schrenk as a ``violator of law,'' understood as an opposition against \emph{law}, \emph{rule}, and \emph{customs}. It conceives lawlessness as \emph{legal injustice}, \emph{partiality in judgment}, \emph{unjust rule}, \emph{dishonesty in business}, and \emph{unlawful actions}, which lead to \emph{incompetence}, what is \emph{unjustifiable}, \emph{harm}, and \emph{injury}. It perceives as \emph{calumnious} and \emph{inimical}, but simply it violates what is \emph{socially acceptable}, \emph{opposes the ethical}, and is \emph{uncivilized}. \emph{Righteousness} defines as ``morally correct behavior, or a feeling that you are behaving in a morally correct way.''\cdfoot{righteousness}{2023-03-05}
Found in Rom 1:29.
