\item[Abuser,]
\entlbl{abuser}

\grc{πλήκτης}
\index[grc]{πληκτης@\grc{πλήκτης}}
(\textit{plēktēs}):
Thayer denotes ``(\grc{πλήσσω}) \ldots, (A. V. striker), bruiser, ready with a blow; a pugnacious, contentious, quarrelsome person.'' Then Gingrich denotes distinctly ``pugnacious man, bully.'' Danker connotes to G. ``[\grc{πλήσσω}; `smiter'] `one who is set on getting into a fight', bully.'' Then when checking an array of cognates against Liddell, the following is denoted: \grc{πλήσσω} \emph{sting}, \emph{strike}, \emph{stamp} as one does a coin, \emph{receive a blow}, \emph{to be smitten} emotionally, \grc{πληκ-τέον} \emph{one must strike}, \grc{πλῆκ-τρον} \emph{instrument for striking the lyre}, \emph{plectrum}, \emph{spear-point}, of  lightning, a bee's \emph{sting}, finally \grc{πλήκ-της} \emph{striker}, \emph{brawler}, \emph{violent}, \emph{fierce}. The total sense here is, punching or beating someone, as in physical and potentially emotional abuse. Thereby looking at the possible person doing so, which supposedly can be called a \emph{brawler}, \emph{bully}, \emph{bruiser}, or understood best as an \emph{abuser}. \emph{Bullying} defines as ``the behavior of a person who hurts or frightens someone smaller or less powerful, often forcing that person to do something they do not want to do,''\cdfoot{bullying}{2023-03-24} and \emph{abusive} defines as ``using physical violence or emotional cruelty.''\cdfoot{abusive}{2023-03-24}
Found in 1~Tim 3:3; Titus 1:7.
