\item[Absurd speech,]
\entlbl{absurd speech}

\grc{μωρολογία}
\index[grc]{μωρολογια@\grc{μωρολογία}}
(\textit{mōrologia}):
According to Bertram (TDNT 4:832), \grc{μωρός} first denotes ``\emph{moros} is related to the Sanskr. \emph{muras} (`dull-witted') and the Indo-European root mo[u]ro-. muro-. We also find the Attic \emph{moros}. The Lat. \emph{morus}, `foolish,' `absurd,' derives from the Gk. \ldots \emph{moros} and cognates denote a physical or intellectual deficiency in animals or men, in their conduct and actions, also in things. The word can refer to physical sloth or dullness, but its main ref. is to the intellectual life,'' later \grc{μωρολογία} denotes ``[844 4.C] Among the warnings in Eph 5 there is found in v4 that against \emph{morologia}. The word occurs alongside \emph{eutrapelia} [\entref{sex-slander}], which means adroitness of speech both in the good sense and the bad. The reference is thus to sins of the tongue, \ldots Hence one is to think in terms of offensive, equivocal and foolish speech. But the terms are taken up yet again in the \ldots of v6, and this seems to carry a warning against heresy. \ldots Jesus and early Christian exhortation are opposed to the empty verbal debates which are found in Jewish schools and also in Gnosticising Christianity \ldots These are described as fools, \ldots They are \ldots foolish questions, like the problem whether Lot's wife as a pillar of salt, or one who has risen from the dead, will make unclean according to the uncleanliness through contact with the dead.'' It is serious talk about serious things but on the level of an imbecile, whether what religiously applies when the power and grace of God can resurrect the dead into life. \emph{Imbecile} defines as the level of intelligence corresponding to ``a person who behaves in an extremely stupid way.''\cdfoot{imbecile}{2023-03-23} Also \emph{absurd} defines as ``ridiculous or completely unreasonable.''\cdfoot{absurd}{2023-03-23}
Found in Eph 5:4.
