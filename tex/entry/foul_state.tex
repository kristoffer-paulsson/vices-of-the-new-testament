\item[Depravement,]
\entlbl{depravement}

\grc{ἀκάθαρτος}
\index[grc]{ακαθαρτος@\grc{ἀκάθαρτος}}
(\textit{akathartos}):
In LXX the term is about the bio-hazards only. See \entref{depravity} for an explicit understanding. In Zech 13:2 there is a promise that God will remove the spirit of depravement and the false prophets out of the land of Israel, also the spiritual Heavenly Kingdom, therefore no depraved person (Eph 5:5) that is an idolater will enter in. The unclean spirits mentioned in the NT 23 times is a spirit demon of depravement which enters through a vile lifestyle. According to Liddell the cognate \grc{ἀκαθαρτίζομαι} means ``\emph{to be ceremonially unclean},'' then \grc{ἀκάθαρτος} denotes ``A. \emph{uncleansed}, \emph{foul}, \ldots of the body, \ldots of a woman, \ldots of ceremonial impurity, \ldots b. \emph{unpurified}, \ldots 2. \emph{morally unclean}, \emph{impure}, \ldots 3. of things, \emph{not purged away}, \emph{unpurged}, \ldots b. \emph{unpruned}, \ldots c. \emph{ceremonially unclean}, of food, \ldots d. \emph{not sifted}, \emph{containing impurities}, \ldots II. Act., \emph{not fit for cleansing}.'' Now the difference between impure and cermonial unclean is the difference in how you treat fellow women.
Found in Eph 5:5.
