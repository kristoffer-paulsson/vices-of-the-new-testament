\item[Disunity,]
\entlbl{disunity}

\grc{διχοστασία}
\index[grc]{διχοστασια@\grc{διχοστασία}}
(\textit{dichostasia}):
\newglossaryentry{dichostasia}
{
    name=\grc{διχοστασία},
    description={\entrefgls{disunity}},
    sort=διχοστασια@\grc{διχοστασία}
}
In order to understand the term, we need to understand its root. 
According to Schlier, \grc{ἀφίστημι} denotes as ``In Heb 3:12 it is used expressly of religious decline from God. \ldots This apostasy entails an unbelief which abandons hope. According to 1 Tim 4:1 apostasy implies capitulation to the false beliefs of heretics. This apostasy is an eschatological phenomenon.''\bkfoot{\grc{ἀφίστημι}}{1:513}{\tdntSchlier{}} 
That said is the beginning of the process of apostasy. The current term, on the other hand, is denoted by S. as ```Division', `disunity', `contention' \ldots In the NT it signifies `objective disunity' in the community. \emph{dichostasia} has a limited `political' sense. It is within the \emph{ekklēsia} that \emph{dichostasia} arises.''\bkfoot{\grc{διχοστασία}}{1:514}{\tdntSchlier{}} 
It's understood to be those people that weigh between two or more doctrinal heresies of misinformation that are misleading to the flock. \emph{Disagreement} is described with ``a situation in which people have different opinions, or an inability to agree,''\cdfoot{disagreement}{2023-03-14} and \emph{contention} using ``disagreement resulting from opposing arguments,''\cdfoot{contention}{2023-03-14} while \emph{disunity} with ``a situation in which people disagree so much that they can no longer work together effectively.''\cdfoot{disunity}{2023-03-14} See also \entref{sectarianism}.
Found in Gal 5:20.
