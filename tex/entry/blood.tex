\item[Blood,]
\entlbl{blood}
 
\grc{αἷμα}
\index[grc]{αιμα@\grc{αἷμα}}
(\textit{haima}):
\newglossaryentry{haima}
{
    name=\grc{αἷμα},
    description={\entrefgls{blood}},
    sort=αιμα@\grc{αἷμα}
}
Behm says that ``the basic physiological meaning is `blood'.''\bkfoot{\grc{αἷμα}}{1:172}{\tdntBehm{}} 
Liddell also points to ``\emph{streams of blood}, \ldots \emph{bloodshed}, \ldots \emph{blood-relationship}.'' 
Gingrich adds ``\emph{bloody deed} \ldots As a means of puricifation'' Then, Danker adds in ``Blood of animals used for cultic purposes.'' 
All in contrast to the shed blood of Jesus Christ for our sins and salvific use. Thus Christians are to abstain and keep from bloody use except for slaughter in meat production by pouring it straight out in the drain and nothing else. Because the life is in the blood and dedicated for the use at the altar, thus it is prohibited to eat (Lev 17:11-14). \emph{Blood} defines as “the red liquid that is sent around the body by the heart, and carries oxygen and important substances to organs and tissue, and removes waste products.”\cdfoot{blood}{2023-03-05}
Found in Acts 15:20, 29, 21:25.
