\item[Blasphemy,]
\entlbl{blasphemy}

\grc{βλασφημία}
\index[grc]{βλασφημια@\grc{βλασφημία}}
(\textit{blasphēmia}):
The word group, according to Beyer, means 
``[A.] In secular Gk. \emph{blasphēmia} is a. `abusive speech' (misuse of words) in contrast to \emph{euphēmia} \ldots J. Wackernagel translates \emph{blasphēmia} as a `word of evil sound,' b. The word means further the strongest form of `personal mockery and calumniation.´ It almost amounts to the same as \emph{loidorein} \ldots c. It means `blasphemy of the deity' by mistaking its true nature of violating or doubting its power. \ldots [B.] The root of \emph{blasphēm-} in the LXX has nothing clearly corresponding in the original. \ldots As distinct to from these synonyms, \emph{blasphēm-} always refers finally to God, whether in the sense of the disputing of His power, \ldots the desecrating of His name by the Gentiles who capture and enslave His people (Isa 52:5), the violation of His glory by derision of the mountains of Israel (Ezek 35:12) and His people, \ldots all ungodly speech and action, especially on the part of the Gentiles, \ldots or human arrogance with its implied depreciation of God \ldots [C.][1.] In the NT the concept of blasphemy is controlled throughout by the thought of violation of the power and majesty of God. Blasphemy may be directed immediately against God, \ldots against the name of God, \ldots against the Word of God (Titus 2:5), against Moses God and therefore against the bearer of the revelation in the Law (Acts 6:11). \ldots [2.] Om the other hand for the Christians it is blasphemy to throw doubt on the lawful Messianic claim of Jesus, to deride Christ in His unity with the Father and as the Bearer of divine majesty. \ldots The fate of being slandered and attacked in their basic faith from Christ to His community in its union with the Lord.''\bkfoot{\grc{βλασφημία}}{1:621--25}{\tdntBeyer{}} 
For personal mockery, the denoted term is weaker than \entref{defamatory}, but mostly blasphemy is generally against God and then spills over on his disciples in him. \emph{Mockery} defines as ``the act of mocking someone or something,''\cdfoot{mockery}{2023-03-06} and \emph{blaspheme} as ``to use offensive words or make statements that show no respect for God or religion.''\cdfoot{blaspheme}{2023-03-06} It's usually translated as \emph{blasphemy} when used against God but as \emph{mockery} when against humans.
Found in Matt 15:19; Mark 7:22; Eph 4:31; Col 3:8.
