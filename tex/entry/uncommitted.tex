\item[Uncommitted,]
\entlbl{uncommitted} 

\grc{ἀσύνθετος}
\index[grc]{ασυνθετος@\grc{ἀσύνθετος}} 
(\textit{asynthetos}):
\newglossaryentry{asynthetos}
{
    name=\grc{ἀσύνθετος},
    description={\entrefgls{uncommitted}},
    sort=ασυνθετος@\grc{ἀσύνθετος}
}
From Liddell, we understand the term denoted as ``\emph{uncompounded}, \ldots a word \emph{standing alone}, \ldots \emph{bound by no covenant}, \emph{faithless}, \ldots \emph{making no covenants}.''
Then: examining the cognates found in L. gives a broader understanding of what the concept of \grc{ἀσύνθ-} is,
\grc{ἀσυνθεσία} ``\emph{breach of covenant}, \emph{transgression}, \ldots \emph{being uncompounded} or \emph{uncombined},''
\grc{ἀσυνθετέω} ``\emph{break covenant}, \emph{be faithless},''
\grc{ἀσύνθηκος} ``\emph{through breach of contract},''
Technically the opposite speaks about two parties being in a covenant, which may also care for each other. 
Thayers generally agree in all. Gingrich adds \emph{faithless},  \emph{untrustworthy}, and \emph{undutiful}. 
Used mainly in the LXX, the denoted \grc{ἀσύνθ-} group of words is mostly about being unfaithful such as in breaking or disregarding the covenant of God, a few times is it in a man/woman marital context.
The term is different from \grc{ἄπιστος} \entref{unfaithful} in such a way that the uncommitted don't regard a covenant, while the unfaithful are not trustworthy as in putting confidence in them. 
\emph{Disconnected} defines as ``separate from someone or something else, and not fitting well together or understanding each other.''\cdfoot{disconnected}{2023-03-12} And, describes \emph{uncommitted} as ``having made no promise to support any particular group, plan, belief, or action.''\cdfoot{uncommitted}{2023-03-12} Also, describes \emph{unattached} as ``not feeling connected to a person, group, or idea.''\cdfoot{unattached}{2023-03-12}
Found in Rom 1:31.
