\item[Uncommitted,]
\entlbl{uncommitted} 

\grc{ἀσύνθετος}
\index[grc]{ασυνθετος@\grc{ἀσύνθετος}} 
(\textit{asynthetos}):
From Liddell, first we want to understand \grc{συντίθημι} (A.III), which in a social context denotes ``commit to a person's care, deliver to him for his own use or that of others,'' we also want an understanding of \grc{σύγκειμαι} (A.III) that socially connotes ``to be agreed on by two parties, \ldots according to what had been agreed on with him.'' Technically the opposite speaks about two parties being in a covenant, which may also care for each other. Then L. says about \grc{ἀσύνθετος} \emph{uncompounded} and \emph{bound by no covenant}, \emph{faithless}. Thayers generally agree in all. Gingrich adds \emph{faithless},  \emph{untrustworthy}, and \emph{undutiful}. About someone that can not blend with someone else in agreement or covenant because of no affection or affiliation with each other -- they don’t like each other -- and it’s obvious. Predominantly used in the Greek bible between husbands and wives, never in between humans and God, compared to \grc{ἄπιστος} \entref{unfaithful}. \emph{Disconnected} defines as ``separate from someone or something else, and not fitting well together or understanding each other.''\cdfoot{disconnected}{2023-03-12} And, describes \emph{uncommitted} as ``having made no promise to support any particular group, plan, belief, or action.''\cdfoot{uncommitted}{2023-03-12} Also, describes \emph{unattached} as ``not feeling connected to a person, group, or idea.''\cdfoot{unattached}{2023-03-12}
Found in Rom 1:31.
