\item[She-devil,]
\entlbl{she-devil} 

\grc{διάβολος}
\index[grc]{διαβολος@\grc{διάβολος}} 
(\textit{diabolos}):
According to Foerster (TDNT 2:71--81), said meaning in Greek is ``have the sense of `separating' \ldots `to separate from.' Hence the passive: `to be set in opposition to someone,' `to hate or be hated by him' \ldots This leads quite easily to the sense of `to accuse' as in the orator Antiphon \ldots In the first instance, of course, \emph{diaballein} does not mean judicial accusation, but the hostile will expressed in complaints and reproaches, and therefore denunciation. \ldots Often, however, it is hard to distinguish, between \emph{diaballein} and `to calumniate' \ldots This leads us to many further meanings. e.g., `to repudiate' \ldots `to misrepresent' \ldots `to give false information' \ldots and esp. `to deceive.''' F. mentions in the NT ``In 1~Tim 3:11 \ldots and 2~Tim 3:3 \ldots the word has the predominant Gk. sense of `calumniator.''' Predominantly in contect about women in general as well as female deacons. Nothing of being a devil here relates to satan as an entity but human behavior and conduct. \emph{Calumny} defines as ``(the act of making) a statement about someone that is not true and is intended to damage the reputation of that person,''\cdfoot{calumny}{2023-03-11} also \emph{devilish} as ``evil or morally bad,''\cdfoot{devilish}{2023-03-14} but \emph{devil} as ``a person who enjoys doing things people might disapprove of,''\cdfoot{devil}{2023-03-14} and \emph{she-devil} as ``a woman who is considered to be dangerous or evil.''\cdfoot{she-devil}{2023-03-14}
Found in 1~Tim 3:11; 2~Tim 3:3.
