\item[False assertion,]
\entlbl{false assertion}

\grc{ψευδομαρτυρία}
\index[grc]{ψευδομαρτυρια@\grc{ψευδομαρτυρία}}
(\textit{pseudomartyria}):
According to Strathmann (TDNT 4:513), the term denotes ```the false witness' \ldots The construction is not to be explained in the same way as \ldots etc., where \emph{pseudo-} implies that what the main word denotes is claimed only \emph{pseudos} [\entref{falsehood}] or falsely.  On the contrary, the main word is taken verbally as in \emph{pseudangelos} (declaring lies, false messenger, Hom., Aristot.) \ldots The word thus means one who attests something which is false. It is not contested that the person concerned is a witness, as though he had no direct knowledge of the persons, relations or events at issue. What is disputed is the correctness of what he says. \ldots When in Plat. Gorg. (472b) in the discussion of the thesis of Socrates \ldots, Polos brings against Socrates the whole host of the Athenians as witnesses, and Socrates calls these \emph{pseudomartyras} with whose help \ldots, he is not saying that they cannot be regarded as witnesses, but simply that \ldots because that they testify to what is false. Whether a man is \emph{martys} or \emph{pseudomartys} depends on whether or not he tells the truth. The biblical use is similar.'' We here talk about someone that is not necessarily the eye or earwitness but someone that could be a herald or messenger, whether spreading information or making statements or claims. It is necessary to certify whether it is absolutely true or if such an impression is delivered. It can easily lead to an accusation of falsely certifying that a said statement isn't true. \emph{assertion} defines as ``a statement that you strongly believe is true.''\cdfoot{assertion}{2023-03-29} See \entref{falsehood}, \entref{liar}.
Found in Matt 15:19.
