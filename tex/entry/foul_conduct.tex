\item[Foul,]
\entlbl{foul}

\grc{ἀκαθαρσία}
\index[grc]{ακαθαρσια@\grc{ἀκαθαρσία}}
(\textit{akatharsia}):
In LXX the term is about the hazardous and contagious but in the Pentateuch only about bio-hazards, also in the TDNT, Hauck describes the New Testament version of filth in terms that denotes the purification through sanctification. (TDNT 3:428) ``uses the words of moral impurity which excludes man from fellowship with God (opposite \emph{hagios}). Paul adopts \emph{akatharsia} from Judaism as a general description of the absolute alienation from God in which heathenism finds itself. But for him the term no longer has ritual significance.'' Liddell denotes filth as \emph{uncleanliness}, the \emph{foulness} of wound, \emph{dirt}, \emph{filth}, in a moral sense \emph{depravity}. Danker agrees with \emph{filth}, \emph{dirt}, and \emph{moral depravity}. It's clearly about filthy/foul conduct, especially in the sphere of economics and social affairs, and may extend somewhat into the sexual arena, even as demonically influenced but mostly as a willful choice of lifestyle. \emph{foul} defines as ``extremely unpleasant,'' and ``to pollute something or make it dirty,'' including ``to commit a foul against another player.''\cdfoot{foul}{2023-03-06}
Found in Gal 5:19; Eph 5:3; 2~Cor 12:21; Col 3:5.
