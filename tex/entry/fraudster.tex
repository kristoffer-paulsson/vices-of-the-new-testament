\item[Fraudster,]
\entlbl{fraudster}

\grc{κλέπτης}
\index[grc]{κλεπτης@\grc{κλέπτης}}
(\textit{kleptēs}):
Preisker (TDNT 3:754--5), first explains \grc{κλέπτω}, and denotes ``a. `To steal,' `secretly and craftily to embezzle and appropriate,' \ldots No blame is attached in these passages; indeed, cunning and skill displayed are recognized, hence gods, demigods and heroes steal (\ldots deduces from Epicurean ethics that stealing is justifiable for this philosophy so long as is takes place \ldots `with craft and secrecy'). Later it's condemned as no less wrong than robbery, murder, and other serious offenses. \emph{klepto} denotes the secret and cunning act as compared with \emph{arpazo}, which is characterized by violence (\emph{bia}) \ldots The objects may be articles of value, \ldots animals, \ldots or men (in the sense `to abduct'), \ldots The ref. might also be to places, \ldots (`to seize with cunning, unnoticed') or to circumstances, \ldots (`to provide for oneself surreptitiously'). b. More generally the word can mean `to deceive,' `to cheat,' `to bewitch (by flattery),' \ldots c. A further meaning is `to hold secretly,' `to put away,' `to conceal,' `to hide.' \ldots d. `To do something in secret or furtive manner.''' Then \grc{κλέπτης} itself denotes as ``is a. `the theif,' \ldots is also b. `one who acts with subterfuge and secrecy,' \ldots `you are found as one who secretly stole the voices which otherwise would have been for him.''' Therefore it could be anything between \emph{abducting} defined as ``to take a person away by force,''\cdfoot{abducting}{2023-03-20} then \emph{thievery} described as ``the crime of stealing things,''\cdfoot{thievery}{2023-03-20} also \emph{trickery} described as ``the use of tricks intended to deceive, as a way of cheating someone,''\cdfoot{trickery}{2023-03-20} and \emph{fraud} defined as ``the crime of obtaining money or property by deceiving people.''\cdfoot{fraud}{2023-03-11}
Found in 1~Cor 6:10.
