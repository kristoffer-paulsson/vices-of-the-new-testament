\item[Gossipmonger,]
\entlbl{gossipmonger} 

\grc{ψιθυριστής}
\index[grc]{ψιθυριστης@\grc{ψιθυριστής}}
(\textit{psithyristēs}):
Liddell connotes ``\emph{a whisperer}: \emph{a slanderer},'' then Thayer says ``\emph{a whisperer}, \emph{secret slanderer}, \emph{detractor},'' and Gingrich mention ``\emph{whisperer}, \emph{talebearer},'' then also Danker says ``gossipmonger, tale-bearer.'' Simply, said, someone slandering behind people's backs without letting them know. The same thing as a slanderer except in secret conversation behind someone's knowledge. \emph{Gossipmonger} defines as ``someone who enjoys talking about other people and their private lives.''\cdfoot{gossipmonger}{2023-03-29}
Found in Rom 1:29.
