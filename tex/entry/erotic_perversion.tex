\item[Erotic perversion,]
\entlbl{erotic perversion}

\grc{πάθος}
\index[grc]{παθος@\grc{πάθος}}
(\textit{pathos}):
According to Michaelis, the term denotes ``1. \emph{pathos}. used from the tragic poets, \ldots It first denotes an `experience,' \ldots Even without addition it is used \emph{in malam partem} for `misfortune,' `mishap,' `defeat,' `sickness,' etc. The meaning `mood,' `feeling,' `emotion' etc. is very common in both good sense and a bad; \ldots This meaning is often \emph{in malam partem}: `passion,' `impulse.' \ldots Under Pythagorean influence is the use of \emph{pathos} for `changes,' `modifications,' `process,' \ldots also `attribute' \ldots As a rhetorical tt. for emotional expression: `pathos.' \ldots 2. In the NT \emph{pathos} occurs only in Pl., plur. Rom 1:26, sing. Col 3:5; 1 Thess 4:5. The \emph{pathe atimias} of Rom 1:26a are the scandalous vices of homosexuality, 1:26b, 27. If \emph{akatharsia}, elsewhere (including Col. 3:5) used for sexual impurity, is more general in 1:24 \ldots, \emph{pathos} is the latest point for transition to the depiction of sexual perversion, and thus denotes erotic passion. When Col 3:5 adds to the demand \ldots the clarifying list: \emph{porneian} [\entref{prostitution}], \emph{akatharsian} [\entref{depravity}], \emph{pathos} [\emph{erotic perversion}], \emph{epithymian kakēn} [\entref{maleficent craving}], appending also the clause \emph{kai tēn pleonexian} [\entref{unscrupulous encroaching}] \emph{hētis estin eidōlolatria} [\entref{idolatry}] \ldots, 4 closely related concepts are contrasted with the final \emph{pleonexia}. Since \emph{porneia} and \emph{akatharsia} \ldots already have a sexual reference, \emph{pathos} here cannot mean `passion' or `feeling' in general, but denotes `erotic passion.' This use leads back to Jewish writings. \ldots, one may detect a contribution made by the Hell. concept. But apart from Philo \ldots we must refer esp. to Joseph. \ldots and Test. Jos 7:8 \ldots esp. as \emph{pathos} and \emph{epithymia ponēra} (= \emph{kakē}, Col 3:5) occur together here, so that \emph{pathos} seems to be an erotic urge which is first given active expression as sin by \emph{epithymia ponēra}. This would give a well-planned climax in Col 3:5 too. Since Rom 1:26 and Col 3:5 are not oriented to the Stoic \emph{pathos} concept, the same applies to 1 Thess 4:5, \ldots \emph{pathos epithymias} does not mean \emph{epithymia} as \emph{pathos} in the Stoic sense but \emph{pathos} as sexual passion which is combined with or grows out of (gen. of origin, not quality as in Rom 1:26) \emph{epithymia} [\entref{craving}].''\bkfoot{\grc{πάθος}}{5:926--28}{\tdntMichaelis{}}
M. denotes that \emph{erotic passion} has a lower boundary where \emph{maleficent craving} begins but not earlier. In modern sense \emph{kink} defines as ``a strange habit, usually of a sexual nature''\cdfoot{kink}{2023-03-11} may not be sinful until a certain limit kicks in. This limit extends from M:s article, which sees a connection between \grc{πάθος} and \grc{ἐπιθυμία κακός}. The conclusion is that sin happens first when kink becomes a) a craving, b) \entref{malevolence}, or c) \emph{maleficent craving}, of which the last also count as idolatry. \emph{Kinky} is also defined as ``unusual, strange, and possibly exciting, especially in ways involving unusual sexual acts,''\cdfoot{kinky}{2023-03-24} and \emph{perversion} as ``sexual behavior that is considered strange and unpleasant by most people.''\cdfoot{perversion}{2023-03-24} Because of modernity and outdated semantic meanings, the term is renamed \emph{erotic perversion}. Thereby being \emph{kinky} below \emph{craving}, \emph{malevolence}, and \emph{maleficent craving}, are not to be counted as \emph{erotic perversion}. 
Found in Col 3:5.
