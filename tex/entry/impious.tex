\item[Impiously wicked,]
\entlbl{impiously wicked} 

\grc{ἀνόσιος}
\index[grc]{ανοσιος@\grc{ἀνόσιος}} 
(\textit{anosios}):
\newglossaryentry{anosios}
{
    name=\grc{ἀνόσιος},
    description={\entrefgls{impiously wicked}},
    sort=ανοσιος@\grc{ἀνόσιος}
}
The term is hard to cover. Liddell mentions \grc{ἀνόσιος} ``\emph{unholy}, \emph{profane}, \ldots of persons, \ldots of things, \ldots a corpse \emph{with all the rites unpaid}, \ldots the holy rites have been \emph{profaned}, \ldots \emph{in unholy wise}, \ldots \emph{without funeral rites}, or \emph{through an unholy deed}.'' When reading L., it is obvious that the term is in relation to heathen conduct and understanding of piety and piousness, not at all a Christian understanding. 
By analyzing the English translation of some Greek texts (citation left out) we try to capture a social understanding of related conduct.
In Herodotus, Histories 2.114 by Godley, it reads as ```A stranger has come, a Trojan, who has committed an \underline{impiety} in Hellas. After defrauding his guest-friend, he has come bringing the man's wife and a very great deal of wealth, driven to your country by the wind'.''\footnote{Herodotus, \emph{Histories} 2.114, trans. Godley.} Used in context about a stranger that conducted fraud, stole a wife, and took the wealth, called impious in regard to heathen gods.
In Herodotus, Histories 3.65 by Godley, it reads as ``Then I feared that my brother would take away my sovereignty from me, and I acted with more haste than wisdom; \ldots but I, blind as I was, sent Prexaspes to Susa to kill Smerdis. \ldots I have killed my brother when there was no need, \ldots So then, the man is dead of an \emph{unholy} destiny at the hands of his relations who ought to have been my avenger for the disgrace I have suffered from the Magi;''\footnote{Herodotus, \emph{Histories} 3.65, trans. Godley.} Used in the context of a Greek king who kills his brother in vanity to keep the throne.
In Antiphon, First Tetralogy 4.7 by Maidment it reads as ``Nay, when can he be cross-examined? He could make a statement in perfect safety; so it is only natural that he was induced to lie about me by his masters, who are enemies of mine. On the other hand, it would be nothing short of \emph{impious} were I put to death by you on evidence which was untrustworthy.''\footnote{Antiphon, \emph{First Tetralogy} 4.7, ed. Maidment.} Contextually used of a man thinking it is impious to kill a slave for lying when he confessed under torture-like conditions.
In Aeschines, On the Embassy 157 by Adams, it reads as ```How outrageous that \ldots but in drunken heat, when Xenodocus, one of the picked corps of Philip, was entertaining us, seized a captive woman by the hair, and took a strap and flogged her'!''\footnote{Aeschines, \emph{On the Embassy} 157, trans. Adams.} Used in the context of a drunken guest at a party grabbing a seized woman and flogging her publicly.
Lots of references regarding pagan values on impiousness are deliberately left out. What purposefully remains is civil impiety, which should make most common people outrageous when they hear about it. There is a very high heathen acceptance of criminal conduct until the unjust death of men or the ravaging of women and plunder. It seems to be worse than \entref{brutal} violence but less than \entref{foolishness}.
 \emph{Impiously} defines as ``in a way that shows no respect, especially for God or religion.''\cdfoot{impiously}{2023-04-25} Also \emph{wicked} as ``morally wrong and bad.''\cdfoot{wicked}{2023-04-25} If someone lives to practice impious wickedness, they should fall out of the Grace of God unto the jurisdiction of the mosaic law. Not until fully repented of the thing, including proven sound before the congregational elders.
Found in 1~Tim 1:9; 2~Tim 3:2.
