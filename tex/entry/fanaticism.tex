\item[Fanaticism,]
\entlbl{fanaticism}

\grc{ζῆλος}
\index[grc]{ζηλος@\grc{ζῆλος}}
(\textit{zēlos}):
\newglossaryentry{zēlos}
{
    name=\grc{ζῆλος},
    description={\entrefgls{fanaticism}},
    sort=ζηλος@\grc{ζῆλος}
}
According to Stumpff, the current term denotes ``is usually translated `zeal,' \ldots a. \emph{zēlos} as the capacity or state of passionate committal to a person or cause is essentially \emph{vox media} \ldots Thus the word is found in Plato in a list with partly good and partly bad emotions \ldots It is also found in the plur. in a biographical and anthropological sense as a comprehensive word to denote the forces which motivate a personality, \ldots also the sense of `taste' or `interest,' \ldots `the warlike spirit (of a tribe),' \ldots Occasionally it can mean `style' in the literary sense, \ldots b. \emph{zēlos} as orientated to a worthy goal \ldots can have the sense of  the `zeal of imitation,' \ldots and this may take on the heightened sense of `passionate rivalry,' \ldots or with only a slight shift of meaning, `zealous recognition,' `praise' or `fame,' \ldots or even `enthusiasm,' \ldots The word here serves  to denote a noble ethical impulse towards the development of character, and to this degree it is to be distinguished from envy or jealousy. \ldots c. Zeal can also take a less reputable form \ldots `passion which poisons human society' \ldots (miserable, pitiable) \ldots (shrieking discordantly, spreading evil rumors).''\bkfoot{\grc{ζῆλος}}{2:877--8}{\tdntStumpff{}}
\emph{Zealot} defines as ``a person who has very strong opinions about something, and tries to make other people have them too,''\cdfoot{zealot}{2023-03-19} and \emph{fanaticism} as ``extreme beliefs that may lead to unreasonable or violent behavior''\cdfoot{fanaticism}{2023-03-19}
Found in Gal 5:20; 2~Cor 12:20; Rom 13:13.
