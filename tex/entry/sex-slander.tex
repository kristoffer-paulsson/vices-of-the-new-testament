\item[Sex-slander,]
\entlbl{sex-slander} 

\grc{εὐτραπελία}
\index[grc]{ευτραπελια@\grc{εὐτραπελία}} 
(\textit{eutrapelia}):
Liddell mentions ``\emph{wit}, \emph{liveliness}, Lat. urbanitas, \ldots 2. in bad sense, \emph{jesting}, \emph{ribaldry}, N.T.'' Then Thayer says ``(from \emph{eutrapelos}, from \emph{eu}, and \emph{trepō} to turn: easily turning; nimble-witted, witty, sharp), \emph{pleasantry}, \emph{humor}, \emph{facetiousness} \ldots in a bad sense, \emph{scurrility}, \emph{ribaldry}, \emph{low jesting} (in which there is some acuteness): Eph 5:4; in a milder sense.'' Also Gingrich mentions \emph{coarse jesting}, \emph{buffoonery}, then Danker mention \emph{suggestive}, \emph{risqué talk}. \emph{Wit} defines as ``the ability to use words in a clever and humorous way,''\cdfoot{wit}{2023-03-19} and \emph{risqué} as ``(of jokes or stories) slightly rude or shocking, especially because of being about sex,'' even \emph{suggestive} as ``often used to describe something that makes people think about sex.''\cdfoot{suggestive}{2023-03-19} The overall feeling is the negative understanding of some sexual undertones and probably very derogatory language in presence of others to be hit or disturbed by such a manner. Sweet talk in front of others for the sake of burning down a relationship, is also probable. \emph{Sexist} defines as ``suggesting that the members of one sex are less able, intelligent, etc. than the members of the other sex, or referring to that sex's bodies, behavior, or feelings in a negative way.''\cdfoot{sexist}{2023-03-19}
Found in Eph 5:4.
