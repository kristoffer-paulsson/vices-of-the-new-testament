\item[Depravity,]
\entlbl{depravity}

\grc{ἀκαθαρσία}
\index[grc]{ακαθαρσια@\grc{ἀκαθαρσία}}
(\textit{akatharsia}):
\newglossaryentry{akatharsia}
{
    name=\grc{ἀκαθαρσία},
    description={\entrefgls{depravity}},
    sort=ακαθαρσια@\grc{ἀκαθαρσία}
}
In the LXX, the term is about the hazardous and contagious. But in the Pentateuch, only about bio-hazards, with an exception for the man who takes his brother's wife (Lev 20:21). In the NT contexts, it is mainly mentioned together with \entref{brutal} violence, \entref{prostitution}, and \entref{unscrupulous encroaching}. Also, the Pharisees were exposed for being filled with \grc{ἀκαθαρσία} because they were constantly stealing each other's wives unscrupulously, just as mentioned in the Law. Liddell mentions ``A. \emph{uncleanness}, \emph{foulness}, of a wound or sore, \ldots b. \emph{dirt}, \emph{filth}, \ldots 2. in moral sense, \emph{depravity}, \ldots 3. \emph{ceremonial impurity}.'' Danker agrees with \emph{filth}, \emph{dirt}, and \emph{moral depravity}. In total, we need to understand that physical filth is no longer an issue, all foods are clean (Mark 7:1--7, 18--19; Cf Acts 10:9), and not washing your hands is no more an issue. Therefore it can only have to do with filth as depravity, doing foul around and against women and the women of brothers. Also, economic filth but apart from the use of women, and at last, normal foul conduct in social affairs. Hauck describes the NT version of filth in terms that denote purification through sanctification. ``uses the words of moral impurity which excludes man from fellowship with God (opposite \emph{hagios}). Paul adopts \emph{akatharsia} from Judaism as a general description of the absolute alienation from God in which heathenism finds itself. But for him the term no longer has ritual significance.''\bkfoot{\grc{ἀκαθαρσία}}{3:428}{\tdntHauck{}} \emph{foul} defines as ``extremely unpleasant,'' and ``to commit a foul against another player.''\cdfoot{foul}{2023-03-06} \emph{depravity} defines as ``the state of being morally bad, or an action that is morally bad''\cdfoot{depravity}{2023-04-24}
Found in Gal 5:19; Eph 5:3; 2~Cor 12:21; Col 3:5.
% Matt 23:27
% Rom 1:23	craving
% Rom 6:19
% 2 Cor 12:21		prostitution, brutality
% Gal 5:19		prostitution, brutality
% Eph 4:19		brutality, unscroupoulus encroachment
% Eph 5:3			prostitution, unscroupoulus encroachment, 
% Col 3:5			prostitution, erotic perversion, maleficent craving, unscroupoulus encroachment
% 1 Thess 2:3		
% 1 Thess 4:7
% brutality x3
% prostitution x4
% unscroupoulus encroachment x3
