\item[Craving,]
\entlbl{craving}

\grc{ἐπιθυμία}
\index[grc]{επιθυμια@\grc{ἐπιθυμία}}
(\textit{epithymia}):
According to Büchsel (TDNT 3:168--171), said term denotes ``the direct impulse towards food, sexual satisfaction etc., and also desire in general. In the first instance there is nothing morally objectionable or even suspicious about them. From the time of Plato, and esp. the Stoics, the term acquires a distinctive sense in Gk. philosophy. \ldots In Plato \emph{epithumia} is still generally \emph{vox media}. Reprehensible desire is called \emph{epithumia kake} [\entref{maleficent craving}]. To true philosophy, however, belongs theoretical and practical aloofness from the sensual world. \ldots Epicurus divides \emph{epithumia} into \emph{phusikai}, or natural, and \emph{kenai}, illegitimate. The first may be divided again into purely natural and those that are necessary to happiness, tp freedom from bodily pain, and to life. \ldots In Greek philosophy \emph{epithumia} is the waywardness of man in conflict with his rationality. It is estimated ethically rather than religiously. \ldots In Hebrew Judaism religion there is condemnation not merely of the evil act but also of the evil will. The Decalogue forbids stealing and the desire for the goods of others, including their wives. \ldots In the NT \emph{epithumia} and \emph{epithumein} are rare in the Gospels, more common in the Epistles. As in current speech, they are often \emph{vox media}. Hence they may be used for the natural desire of hunger, \ldots or longing, \ldots or the desire of the \textbf{divine mysteries}, or for anything good. \ldots Mostly, however, they indicate evil desire in accordance with the Greek and Jewish development considered under A. They may be characterized  as such by information as to the object: \ldots a woman, \ldots or the direction, \ldots or the vehicle, \ldots the world, \ldots or the manner \ldots can be used for sinful desire without any such addition. \ldots For Paul, who alone in the NT offers an explicit doctrine of sinful man, \emph{epithumia} is a manifestation of the sin which  dwells in man and which controls him, but which is dead apart from the \emph{epithumia} stirred up by the Law, Rom 7:7, 8. That desire is a result of the prohibition of sin reveals the carnality of man, Gal 5:16, 24, his separation from God, his subjection to divine wrath, Rom 1:18 ff. In James (1:14, 15) \emph{epithumia} is regarded as the constant root in man of the individual acts of sin which the author's attention is mainly directed. The essential point in \emph{epithumia} is that it is desire as impulse, as motion of the will. It is, in fact, lust, since the thought of satisfaction gives pleasure and that of non-satisfaction pain. \emph{epithumia} is anxious self-seeking. Only exceptionally do we read of an \emph{epithumein} of love; \emph{epithumein} is normally used. In \emph{epithumein} man is seen as he really is, the more so because \emph{epithumia} bursts upon him with the force of immediacy. Even after the reception of the divine spirit [lower case], \emph{epithumia} is always a danger against which man must be warned and must fight.'' We need to understand that there are natural desires and cravings that are impulsive, they will only be overcome through sanctification which is the purpose of the Law to make us discover. Be aware of lusts overtaking your members and rather to the will of the Spirit. \emph{Lust} defines as ``strong desire,''\cdfoot{lust}{2023-03-17} and \emph{craving} as ``a strong or uncontrollable desire.''\cdfoot{craving}{2023-03-17}
Found in Titus 3:3; 1~Pet 4:3.
