\item[Disobedient,]
\entlbl{disobedient}

\grc{ἀπειθής}
\index[grc]{απειθης@\grc{ἀπειθής}}
(\textit{apeithēs}):
\newglossaryentry{apeithēs}
{
    name=\grc{ἀπειθής},
    description={\entrefgls{disobedient}},
    sort=απειθης@\grc{ἀπειθής}
}
The term with its noun form \grc{ἀπείθεια} and verb form \grc{ἀπειθέω} in the New Testament is correct to say only contrary to the obedience of: God, Christ, the Gospel and the stated will of God. According to Bultmann, ``in the sense `disobedient' is found in Gk. from the time of Thucyd. (earlier still in the sense `unworthy of belief').''\bkfoot{\grc{ἀπειθής}}{6:10}{\tdntBultmann{}}
\emph{Disobedient} defines as ``refusing to do what someone in authority tells you to do.''\cdfoot{disobedient}{2023-03-11}
Found in Titus 3:3.
