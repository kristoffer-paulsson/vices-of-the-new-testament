\item[Disinforming,]
\entlbl{disinforming}

\grc{δίλογος}
\index[grc]{διλογος@\grc{δίλογος}}
(\textit{dilogos}):
\newglossaryentry{dilogos}
{
    name=\grc{δίλογος},
    description={\entrefgls{disinforming}},
    sort=διλογος@\grc{δίλογος}
}
According to Liddell \emph{double-tongued}, \emph{doubtful}, in Thayer \emph{double-tongued}, \emph{double in speech}, ``saying one thing with one person, another with another (with intent to deceive),'' Gingrich also mention \emph{double-tongued}, \emph{insincere}. In the context of a congregational deacon. \emph{Disinformation} as ``false information spread in order to deceive people.''\cdfoot{disinformation}{2023-03-11} And \emph{misleading} as ``causing someone to believe something that is not true.''\cdfoot{misleading}{2023-03-11}
Found in 1~Tim 3:8.
