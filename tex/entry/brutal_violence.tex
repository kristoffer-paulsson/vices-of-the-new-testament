\item[Brutal (violence),]
\entlbl{brutal}

\grc{ἀσέλγεια}
\index[grc]{ασελγεια@\grc{ἀσέλγεια}}
(\textit{aselgeia}):
It should be said that general bible translations use words such as \emph{lasciviousness}, and secondly \emph{wantonness}. Another thing must be said, that it is used much differently in Hellenistic Greek. 
Danker opens up that it is unclear, ``[etym. unclear (many conflicting theories), cp. \emph{aselgēs} `licentious'] `wanton disregard for social/moral standards.''' 
Liddell proposes a different interpretation, \grc{ἀσέλγεια} also means ``\emph{wanton violence}, \ldots \emph{insolence},'' 
then: examining the cognates found in L. gives a broader understanding of what the concept of \grc{ἀσέλγ-} is,
\grc{ἀσελγαίνω} ``\emph{behave licentiously}, \ldots \emph{outrageous acts},'' 
\grc{ἀσελγής} ``\emph{licentious}, \emph{wanton}, \emph{brutal}, \ldots generally, \emph{outrageous},''
\grc{ἀσελγό-κερως} ``\emph{with outrageous horn},''
\grc{ἀσελγο-μανέω} ``\emph{to be madly dissolute}.''
By analyzing the Greek text with its English translation, we will find out its real meaning. 
Starting with \grc{ἀσέλγεια}. 
In Plato, Republic 4.424e by Burnet, it reads as ``\grc{τοὺς νόμους καὶ πολιτείας σὺν πολλῇ, ὦ Σώκρατες,} \underline{\grc{ἀσελγείᾳ}}, \grc{ἕως ἂν τελευτῶσα πάντα ἰδίᾳ καὶ δημοσίᾳ ἀνατρέψῃ},''\footnote{Plato, \emph{Republic} 4.424e, ed. Burnet.}
translated by Shorey into ``it proceeds against the laws and the constitution with \underline{wanton licence} [emphasis added], Socrates, till finally it overthrows all things public and private.''\footnote{Plato, \emph{Republic} 4.424e, trans. Shorey.} Used in a discussion about lawlessness in society. 
In Isaeus, Pyrrhus 3.13 by Forster, it reads as ``\grc{οἳ μάχας καὶ κώμους καὶ} \underline{\grc{ἀσέλγειαν}} \grc{πολλήν, ὁπότε ἡ τούτου ἀδελφὴ εἴη παρ᾽ αὐτῷ, μεμαρτυρήκασι γίγνεσθαι περὶ αὐτῆς,}''\footnote{Isaeus, \emph{Pyrrhus} 3.13, ed. Forster.}
translated by Forster into ``who have given evidence of quarrels, serenades, and frequent scenes of \underline{disorder} [emphasis added] which the defendant's sister occasioned whenever she was at Pyrrhus's house.''\footnote{Isaeus, \emph{Pyrrhus} 3.13, trans. Forster.} Used with wild partying at Pyrrhus house with a sister. 
In Demosthenes, Philippic 1 4.9 by Butcher, it reads as ``\grc{ὁρᾶτε γάρ, ὦ ἄνδρες Ἀθηναῖοι, τὸ πρᾶγμα, οἷ προελήλυθ᾽} \underline{\grc{ἀσελγείας}} \grc{ἅνθρωπος},''\footnote{Demosthenes, \emph{Philippic 1} 4.9, ed. Butcher.}
translated by Vince into ``For observe, Athenians, the height to which the fellow's \underline{insolence} [emphasis added] has soared.''\footnote{Demosthenes, \emph{Philippic 1} 4.9, trans. Vince.} Used in a political demagogical speech. 
In Demosthenes, Against Midias 21.1 by Butcher and Rennie, it reads as ``\grc{τὴν μὲν} \underline{\grc{ἀσέλγειαν}}, \grc{ὦ ἄνδρες δικασταί, καὶ τὴν ὕβριν, ᾗ πρὸς ἅπαντας ἀεὶ χρῆται Μειδίας}, ''\footnote{Demosthenes, \emph{Against Midias} 21.1,\\eds. Butcher and Rennie.}
translated by Murray into ``The \underline{brutality and insolence} [emphasis added] with which Meidias treats everyone alike are.''\footnote{Demosthenes, \emph{Against Midias} 21.1, trans. Murray.} Used in a speech about brutalities and assaults in combination with a demagogical speech. 
Then let's examine\grc{ἀσελγής}.
In Andocides, Against Alcibiades 4.40 by Maidment, it reads as ``\grc{τοῦτον δὲ κολάσαντες τοὺς} \underline{\grc{ἀσελγεστάτους}} \grc{νομιμωτέρους ποιήσετε},''\footnote{Andocides, \emph{Against Alcibiades} 4.40, ed. Maidment.}
translated by Maidment into ``you will inspire a greater respect for the law in those whose \underline{insolence} [emphasis added] is uncontrolled.''\footnote{Andocides, \emph{Against Alcibiades} 4.40, trans. Maidment.} Used in a speech about legal punishment. 
In Demosthenes, Olynthiac 2 19 by Butcher, it reads as ``\grc{δῆλον δ᾽ ὅτι ταῦτ᾽ ἐστὶν ἀληθῆ: καὶ γὰρ οὓς ἐνθένδε πάντες ἀπήλαυνον ὡς πολὺ τῶν θαυματοποιῶν} \underline{\grc{ἀσελγεστέρους}} \grc{ὄντας},''\footnote{Demosthenes, \emph{Olynthiac 2} 19, ed. Butcher.}
translated by Vince into ``This report is obviously true, for the men who were unanimously expelled from Athens, as being of far \underline{looser morals} [emphasis added] than the average mountebank.''\footnote{Demosthenes, \emph{Olynthiac 2} 19, trans. Vince.} Used about looser morals leading to being expelled. 
In Demosthenes, Against Midias 21.128 by Butcher and Rennie, it reads as ``\grc{σώφρονα καὶ μέτριον πρὸς τἄλλα παρεσχηκὼς αὑτὸν Μειδίας καὶ μηδένα τῶν ἄλλων πολιτῶν ἠδικηκὼς εἰς ἔμ᾽} \underline{\grc{ἀσελγὴς}} \grc{μόνον οὕτω καὶ βίαιος ἐγεγόνει},''\footnote{Demosthenes, \emph{Against Midias} 21.128,\\eds. Butcher and Rennie.}
translated by Murray into ``Meidias had in other respects behaved with decency and moderation, if he had never injured any other citizen, but had confined his \underline{brutality and violence} [emphasis added] to me.''\footnote{Demosthenes, \emph{Against Midias} 21.128, trans. Murray.} Used in a speech regarding morals and violence.
In Isaeus, Ciron 8.43 by Forster, it reads as ``\grc{οὕτω τοίνυν} \underline{\grc{ἀσελγὴς}} \grc{ὢν καὶ βίαιος καὶ τὴν τῶν ἀδελφῶν οὐσίαν ἀπεστερηκὼς οὐκ ἀγαπᾷ τὰ ἐκείνων ἔχων},''\footnote{Isaeus, \emph{Ciron} 8.43, ed. Forster.} 
translated by Forster into ``This man, then, having shown himself so \underline{brutal and violent} [emphasis added] and having robbed his sisters of their fortune.''\footnote{Isaeus, \emph{Ciron} 8.43, trans. Forster.} Used in the context of robbery. 
In Lysias, On the Refusal of a Pension 24.15 by Lamb, it reads as ``\grc{λέγει δ᾽ ὡς ὑβριστής εἰμι καὶ βίαιος καὶ λίαν} \underline{\grc{ἀσελγῶς}} \grc{διακείμενος},''\footnote{Lysias, \emph{On the Refusal of a Pension} 24.15, ed. Lamb.}
translated by Lamb into ``He says that I am insolent, savage, and \underline{utterly abandoned} [emphasis added] in my behavior.''\footnote{Lysias, \emph{On the Refusal of a Pension} 24.15, trans. Lamb.}  
\grc{ὕβριστος} is also used in context but denotes \emph{outrageous} and \emph{insolence}, therefore its possible that the term specifically denotes \emph{brutal violence}. It is the kind of violence that leads to legal punishment and being expelled. It is impossible that \grc{ἀσέλγεια} has to do with soft sexual or sensual behavior, just because it may be related to drunkenness and carousal sometimes. \emph{Savage} denotes as "extremely violent, wild, or frightening."\cdfoot{savage}{2023-03-11} Also, \emph{violent} as "using force to hurt or attack."\cdfoot{violent}{2023-03-11} Finally, \emph{brutal} as "cruel and violent."\cdfoot{brutal}{2023-03-11}
Found in Gal 5:19; Rom 13:13; Mark 7:22; 2~Cor 12:21; 1~Pet 4:3.
