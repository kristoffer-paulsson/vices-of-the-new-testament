\item[Brutal (violence),]
\entlbl{brutal (violence)}

\grc{ἀσέλγεια}
\index[grc]{ασελγεια@\grc{ἀσέλγεια}}
(\textit{aselgeia}):
It should be said that general bible translations use words such as \emph{lasciviousness}, and secondly \emph{wantonness}. Another thing must be said, that it is used much differently in Hellenistic Greek. Danker opens up that it is unclear, ``[etym. unclear (many conflicting theories), cp. \grc{ἀσελγής} `licentious'] `wanton disregard for social/moral standards.''' Liddell proposes a different interpretation, \grc{ἀσέλγεια} also means \emph{wanton violence},  \emph{insolence}, and \grc{ἀσελγής} also means \emph{brutal}, \emph{outrageous}, and \grc{ἀσελγαίνω} also means \emph{outrageous acts}. By analyzing the Greek text with its English counterpart, we will find out its real meaning. \grc{ἀσέλγεια} translates in Pl.R.424e ``\grc{τοὺς νόμους καὶ πολιτείας σὺν πολλῇ, ὦ Σώκρατες,} \underline{\grc{ἀσελγείᾳ}}, \grc{ἕως ἂν τελευτῶσα πάντα ἰδίᾳ καὶ δημοσίᾳ ἀνατρέψῃ.}'' to ``it proceeds against the laws and the constitution with \underline{wanton licence} [emphasis added], Socrates, till finally it overthrows all things public and private.'' and is used in a discussion about lawlessness in society. In Is.3.13 as ``\grc{οἳ μάχας καὶ κώμους καὶ} \underline{\grc{ἀσέλγειαν}} \grc{πολλήν, ὁπότε ἡ τούτου ἀδελφὴ εἴη παρ᾽ αὐτῷ, μεμαρτυρήκασι γίγνεσθαι περὶ αὐτῆς.}'' to ``who have given evidence of quarrels, serenades, and frequent scenes of \underline{disorder} [emphasis added] which the defendant's sister occasioned whenever she was at Pyrrhus's house,'' used with wild partying. In D.4.9 as ``\grc{ὁρᾶτε γάρ, ὦ ἄνδρες Ἀθηναῖοι, τὸ πρᾶγμα, οἷ προελήλυθ᾽} \underline{\grc{ἀσελγείας}} \grc{ἅνθρωπος,}'' to ``For observe, Athenians, the height to which the fellow's \underline{insolence} [emphasis added] has soared,'' used in a political speech. In Id.21.1 as ``\grc{τὴν μὲν} \underline{\grc{ἀσέλγειαν}}\grc{, ὦ ἄνδρες δικασταί, καὶ τὴν ὕβριν, ᾗ πρὸς ἅπαντας ἀεὶ χρῆται Μειδίας}'' to ``The \underline{brutality and insolence} [emphasis added] with which Meidias treats everyone alike are,'' used in a speech about brutalities and assaults in combination to parties. \grc{ἀσελγής} translates in And.4.40 as ``\grc{τοῦτον δὲ κολάσαντες τοὺς} \underline{\grc{ἀσελγεστάτους}} \grc{νομιμωτέρους ποιήσετε.}'' to ``you will inspire a greater respect for the law in those whose \underline{insolence} [emphasis added] is uncontrolled,'' used in a speech about legal punishment. In D.2.19 as ``\grc{δῆλον δ᾽ ὅτι ταῦτ᾽ ἐστὶν ἀληθῆ: καὶ γὰρ οὓς ἐνθένδε πάντες ἀπήλαυνον ὡς πολὺ τῶν θαυματοποιῶν} \underline{\grc{ἀσελγεστέρους}} \grc{ὄντας}'' to ``This report is obviously true, for the men who were unanimously expelled from Athens, as being of far \underline{looser morals} [emphasis added] than the average mountebank,'' used about looser morals leading to being expelled. In Id.21.128 as ``\grc{σώφρονα καὶ μέτριον πρὸς τἄλλα παρεσχηκὼς αὑτὸν Μειδίας καὶ μηδένα τῶν ἄλλων πολιτῶν ἠδικηκὼς εἰς ἔμ᾽} \underline{\grc{ἀσελγὴς}} \grc{μόνον οὕτω καὶ βίαιος ἐγεγόνει}'' to ``Meidias had in other respects behaved with decency and moderation, if he had never injured any other citizen, but had confined his \underline{brutality and violence} [emphasis added] to me.'' In Is.8.43 as ``\grc{οὕτω τοίνυν} \underline{\grc{ἀσελγὴς}} \grc{ὢν καὶ βίαιος καὶ τὴν τῶν ἀδελφῶν οὐσίαν ἀπεστερηκὼς οὐκ ἀγαπᾷ τὰ ἐκείνων ἔχων}'' to ``This man, then, having shown himself so \underline{brutal and violent} [emphasis added] and having robbed his sisters of their fortune,'' used in the context of robbery. In Lys.24.15 as ``\grc{λέγει δ᾽ ὡς ὑβριστής εἰμι καὶ βίαιος καὶ λίαν} \underline{\grc{ἀσελγῶς}} \grc{διακείμενος,}'' to ``He says that I am insolent, savage, and \underline{utterly abandoned} [emphasis added] in my behavior.'' \grc{ὕβριστος} is also used in context but denotes \emph{outrageous} and \emph{insolence}, therefore its possible that said word specifically denotes \emph{brutal violence}. It is the kind of violence that leads to legal punishment and being expelled. It is impossible that \grc{ἀσέλγεια} has to do with soft sexual or sensual behavior, just because it may be related to drunkenness and carousal sometimes. \emph{Savage} denotes as "extremely violent, wild, or frightening."\cdfoot{savage}{2023-03-11} Also, \emph{violent} as "using force to hurt or attack."\cdfoot{violent}{2023-03-11} Finally, \emph{brutal} as "cruel and violent."\cdfoot{brutal}{2023-03-11}
Found in Gal 5:19; Mark 7:22; 2~Cor 12:21; 1~Pet 4:3.
