\item[Perjured,]
\entlbl{perjured} 

\grc{ἐπίορκος}
\index[grc]{επιορκος@\grc{ἐπίορκος}} 
(\textit{epiorkos}):
\newglossaryentry{epiorkos}
{
    name=\grc{ἐπίορκος},
    description={\entrefgls{perjured}},
    sort=επιορκος@\grc{ἐπίορκος}
}
According to J. Schneider, it states ``to commit perjury,'' as well as ``to swear falsely.'' When it comes to perjury, J. S. states ``1~Tim 1:8 ff. deals with the relation of the Christian and the Law. To the righteous man, i.e. the justified man set in a new life, the requirement of the Law does not apply. Only where sins arise, which are enumerated in a catalog of vices in v9 and v10, are the strictness and severity of the Law relevant. Perjurers are among the enemies of the Law and the gainsayers.''\bkfoot{\grc{ἐπίορκος}}{5:466}{\tdntSchneider{}} 
According to Jesus, we are supposed (Matt 5:33--37) not to swear at all, instead let yes be yes and no be no, accordingly also at liberty to remain silent, not engaging or avoiding. \emph{Perjured} defines as ``false, in a way that involves perjury (= telling lies in court).''\cdfoot{perjured}{2023-03-11} When perjuring so, it is applicable to fall away from the Grace of God into the mosaic jurisdiction if they do not turn away from it.
Found in 1~Tim 1:10.
