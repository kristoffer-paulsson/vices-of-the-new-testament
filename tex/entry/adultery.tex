\item[Adultery,]
\entlbl{adultery}

\grc{μοιχεία}
\index[grc]{μοιχεια@\grc{μοιχεία}}
(\textit{moicheia}):
In traditional catholic and protestant state religion, adultery happens when two persons, a man and a woman, where at least one is married, have intercourse with the one that is not their marital partner. It is not so in either NT or OT. In the OT, a man can divorce his wife if she behaves unseemly (Deut 24:1) or stone her at said intercourse (Lev 20:10) with another man. Thus the NT is sharpened compared to the OT, he can only correctly divorce her if she commits \entref{prostitution}, and he can verify it (Matt 5:32). In the NT, a man commits adultery when scouting after a \emph{gynaika} for the \entref{craving} of her (Matt 5:28). Equally, a woman moving in with another man without being properly divorced or separated is called an adulteress (Rom 7:3). Then if someone divorces, it’s not a sin by itself. Remarriage is adultery under the wrong circumstances, one-sided for the marital dumper and for the man who marries the woman being a marital dumper. When laying out all verses (Matt 5:32, 19:9; Mark 10:11--12; Luke 16:18) alongside and summing up the doctrine, occasionally there will be different results even when using well-translated Bibles such as KJV, YLT, ASV, NABRE, NIV, and similar. Even when not mixing multiple versions in parallel.
Found in Matt 15:19; Mark 7:22.
