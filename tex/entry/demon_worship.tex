\item[Demon worship,]
\entlbl{demon worship}

\grc{προσκυνέω ὁ δαιμόνιον}
\index[grc]{προσκυνεω ο δαιμονιον@\grc{προσκυνέω ὁ δαιμόνιον}}
(\textit{proskyneō ho daimonion}):
According to Foerster (TDNT 2:1--20), ``[A.2] It is first used a. to denote `gods,' and may still be used in the sense in Hellenism. More specifically, it is used b. for `lesser dieties.' This is Plato's allusion when he defines \emph{daimones} as \emph{theoi} \ldots Since \emph{daimon} is more general than \emph{theos}, it is used c. when an `unknown superhuman factor' is at work \ldots Again, especially in the tragic dramatists, it denotes d. `anything that overtakes man,' such as destiny, or death, or any good or evil fortune \ldots It can also be use generally for `fate,' \ldots From this sense it is only a step to e. that of a `protective deity' watching over a man's life, or certain portions of it. \ldots By the time of the Orphics this has led to the coining of the words \emph{eudaimon} and \emph{kakodaimon} [\emph{evil demon}, my translation] \ldots In Stoicism  \emph{daimon} then became f. a term for the `divinely related element in man.''' Further, ``[A.3] In the more detailed development of the doctrine that demons are intermediary beings, regard is had to popular belief at three specific points. [1.] First, it is noteworthy that demons are brought into special connection with those parts of the cultus  and religion which are closest to animism, i.e., with magic incantations. \ldots there emerges already a distinction between the higher forms of religion and the lower and more popular forms with which demons or evil demons are connected. In the developed form of this conception demons are forces which seek to divert from true worship. \ldots [2.] Secondly, it is to be noted that the demons as rulers of human destiny are specifically connected with misfortune and distress. [3.] Thirdly, many philosophical systems have assimilated the doctrine of demons possessing men. Extraordinarily conditions are popularly ascribed to indwelling deities, especially in the tragic dramatists and e.g. Hippocrates. This was called \emph{daimonan} or \emph{daimonizesthai}, a view which is developed \ldots to the effect that evil demons clothe themselves with flesh and blood in the human body to kindle evil desires. But Plutarch already speaks plainly of demons which undermine virtue. \ldots this view is then linked with astrology \ldots and on the other hand it can be argued that falsehood belongs to the very essence of demons. \ldots Philosophy incorporated these intermediaries  into its system and world view by ascribing \emph{pathe} to demons and by giving at least to evil demons a location close to the earth. \ldots Their wickedness is not simply that of implacably and causelessly evil will; it is due to their link with matter, and may thus be regarded as an impulsion by cravings which are not familiar to man, whether in the form of envy, or a self-seeking desire for honor, or the thirst for blood and the odor of sacrifice.'' Finally, in the NT ``[C.2] Basically the NT stands in the OT succession. There is no reference to spirits of the dead; the dead sleep until the resurrection. \emph{Daimon}, with its suggestion of an intermediary between God and man, is avoided. Angels and demons are antithetical. Indeed, it is only in the NT that we have a full and radical distinction. \ldots When we survey the passages mentioned, we note first how comparatively infrequent are the NT references to demons except in the case of the possessed. No trace whatever remains of the belief in ghosts, which is so important in the Rabbis. \ldots Nevertheless, the fact that demons are mentioned only with relative infrequency in the NT does not mean that their existence and operation are contested or doubted. For Paul witchcraft is meddling with demons. But there can also be intercourse with demons in the normal heathen cultus (1 Cor 10:20 f.).'' \emph{Worship} defines as ``to have or show a strong feeling of respect and admiration for God or a god,'' also ``to go to a religious ceremony.''\cdfoot{worship}{2023-03-13} Finally, \emph{demon} as ``an evil spirit.''\cdfoot{demon}{2023-03-13}
Found in Rev 9:20.
