\item[\sout{Abominable},] renamed to \emph{abhorrers}.
\item[Abhorrer,]
\entlbl{abhorrer}

\grc{βδελύσσομαι}
\index[grc]{βδελυσσομαι@\grc{βδελύσσομαι}}
(\textit{bdelyssomai}):
\newglossaryentry{bdelyssomai}
{
    name=\grc{βδελύσσομαι},
    description={\entrefgls{abhorrers}},
    sort=βδελυσσομαι@\grc{βδελύσσομαι}
}
According to Foerster ``the basic stem with its sense of causing abhorrence. \emph{bdelyssomai ktl}. \emph{bdelyros} and its derivatives
\emph{bdelyreuomai} and \emph{bdelyria} are often found in the secular field to denote an improper attitude, often in connection with such related expressions
as \emph{anaischyntos}, \emph{miaros}, \emph{thrasys}. In particular this word group denotes a shameless attitude.''
About the lexeme \grc{βδελύσσομαι}, F. writes ``is a middle pass. with acc. in the sense of `to loathe,' `to abhor,' though it later takes on the more intensive
meaning of `to censure' or `to reject'.''\bkfoot{\grc{βδελύσσομαι}}{1:5}{\tdntFoerster{}}
In this specific case we are looking for the use in Rev 21:8 which according to BNM is \grc{ἐβδελυγμένοις}\footnote{1 Cor 6:9; word ``7. \grc{ἐβδελυγμένοις βδελύσσω} -
vpxmdmp (verb participle perfect middle dative masculine plural),'' (BNM) \emph{BibleWorks 10.0.8.755} BibleWorks, 2017.},
here it is in perfect middle participle which indicates a) that it is used as an adjective and b) that the aspect should read as ``I have been.'' Which means that the sins has happened
in the past, this is because the biblical context is in the future and refer back to those who commited sins before. According to Liddell which is a bit stricter in its interpretation
it could also mean \emph{cause to stink}, \emph{make loathsome/abominable}, in this interpretation it is only those who practice to spread a bad reputation that are affected.
Therefore the true meaning should come to ``they have caused abhorrence (against someone).'' CD defines \emph{abhorrence} as ``a feeling of hating something or someone.''
Found in Rev 21:8.