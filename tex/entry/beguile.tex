\item[Beguile,]
\entlbl{beguile}

\grc{ψεύδομαι}
\index[grc]{ψευδομαι@\grc{ψεύδομαι}}
(\textit{pseudomai}):
The term is a sin in the context of the brethren in Christ, not concerning the world. Liddell mentions ``\emph{to lie}, \emph{speak false}, \emph{play false}, \ldots \emph{to say that which is untrue}, \ldots which \emph{I do} not \emph{speak falsely} about him, \ldots \emph{to be false}, \emph{perjured} or \emph{forsworn}, \ldots \emph{to belie}, \emph{falsify}, \ldots \emph{to break} them, \ldots \emph{had broken their word} about the money, \ldots \emph{to deceive by lies}, \emph{cheat}, \ldots \emph{to deceive} one \emph{in} a thing.'' Then Thayer also mentions ```to deceive', `cheat'; hence, properly, \emph{to show oneself deceitful}, \emph{to play false} \ldots \emph{to lie}, \emph{to speak deliberate falsehoods} \ldots \emph{to deceive one by a lie}, \emph{to lie to}.'' The context totally is not to infiltrate the Body of Christ, nor to be a false brother, but mostly to be honest and worthy of fellowship in the Body. \emph{Beguile} is defined as ``to persuade, attract, or interest someone, sometimes in order to deceive them,''\cdfoot{beguile}{2023-03-29} see \entref{falsehood}, \entref{deceiving}, and \entref{liar}.
Found in Col 3:9.
