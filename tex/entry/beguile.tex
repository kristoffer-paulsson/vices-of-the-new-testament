\item[Beguile,]
\entlbl{beguile}

\grc{ψεύδομαι}
\index[grc]{ψευδομαι@\grc{ψεύδομαι}}
(\textit{pseudomai}):
The term is a sin that contextually applies to brethren in the Body of Christ, not concerning the world. ``to lie, speak false, play false, \ldots to say that which is untrue, \ldots which I do not speak falsely about him, \ldots to be false, perjured or forsworn, \ldots to belie, falsify, \ldots to break them, \ldots had broken their word about the money, \ldots to deceive by lies, cheat, \ldots to deceive one in a thing.'' Then Thayer also denotes ```to deceive', `cheat'; hence, properly, to show oneself deceitful, to play false \ldots to lie, to speak deliberate falsehoods \ldots to deceive one by a lie, to lie to \ldots like verbs of saying.'' Then Danker and Gingrich shortly repeats about not living in falsehood to one another. \emph{Beguile} defines as ``to persuade, attract, or interest someone, sometimes in order to deceive them,''\cdfoot{beguile}{2023-03-29} see \entref{falsehood}, \entref{deceiving}, and \entref{liar}.
Found in Col 3:9.
