\item[Malevolence,]
\entlbl{malevolence} 

\grc{κακία}
\index[grc]{κακια@\grc{κακία}} 
(\textit{kakia}):
To understand \grc{κακία} we first much understand \grc{κακός}. 
According to Grundmann, \grc{κακός} denotes as ``The word \emph{kakos}, already considered in relation to \emph{agathos}, expresses the presence of the lack. It is not positive; it is an incapacity or weakness. Like `evil,' it has more than purely moral significance. The wealth expressed in the developing concepts \ldots Thus \emph{kakos} means a. `mean,' `unserviceable,' `incapable,' `poor of its kind,' \ldots Greater precision is attained by additions. \ldots It also means b. `morally bad,' `wicked.' \ldots It then means c. `weak.' \ldots A final meaning d. is `unhappy,' `bad,' `ruinous,' `evil.' \ldots This fixes the meaning of the noun \emph{to kakon}, \emph{ta kaka}, `evil,' `suffering,' `misfortune,' `ruin.'''\bkfoot{\grc{κακός}}{3:469}{\tdntGrundmann}  
Maleficent as a concept based on being mean and harmful by ignorance and incompetence causing suffering and ruin. \emph{Maleficent} defines as ``bad or harmful.''\cdfoot{maleficent}{2023-03-19} 
The said word \grc{κακία}, then according to G. denotes ``This word is related to \emph{kakon} \ldots It is the quality of a \emph{kakos}, and it can also signify the outworking of this quality, sometimes in the plural.''\bkfoot{\grc{κακία}}{3:482}{\tdntGrundmann}
\emph{Malevolence} defines as ``the quality of causing or wanting to cause harm or evil.''\cdfoot{malevolence}{2023-03-19}
See \entref{maleficent contriver}, \entref{maleficent craving}, \entref{maleficent eye}, also \entref{malice}.
Found in Rom 1:29; Eph 4:31; Titus 3:3; Col 3:8; 1~Pet 2:1.
