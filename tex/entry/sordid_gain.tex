\item[Sordid gain,]
\entlbl{sordid gain} 

\grc{αἰσχροκερδής}
\index[grc]{αισχροκερδης@\grc{αἰσχροκερδής}} 
(\textit{aischrokerdēs}):
Stems from two words: \grc{αἰσχρός} \emph{sordid}, and \grc{κέρδος}, which according to Schlier (TDNT 3:672) means \emph{to gain}, \emph{advantage}, \emph{profit}, also \emph{crafty counsels} and \emph{cunning}. Further, Liddell states: \emph{sordidly greedy for gain}, and Thayer states: \emph{from eagerness for base gain}. In TDNT, it is clear that it is not limited strictly to the economic sphere but rather \emph{gain} in general. Sordid \emph{gain} defines as ``to get something that is useful, that gives you an advantage, or that is in some way positive, especially over a period of time.''\cdfoot{gain}{2023-03-05} Also, see \entref{sordid}. Simply shameful gain sordidly.
Found in 1~Tim 3:8; Titus 1:7.
