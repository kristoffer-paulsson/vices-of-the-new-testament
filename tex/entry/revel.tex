\item[Revel,]
\entlbl{revel} 

\grc{κῶμος}
\index[grc]{κωμος@\grc{κῶμος}} 
(\textit{kōmos}):
All Liddell, Thayer, and Gingrich basically agree on \emph{revelry} and \emph{carousal}. By analyzing the contexts given in L. we find: 
In Herodotus, Histories 1.21 by Godley, it reads as ``\grc{ὅσος ἦν ἐν τῷ ἄστεϊ σῖτος καὶ ἑωυτοῦ καὶ ἰδιωτικός, τοῦτον πάντα συγκομίσας ἐς τὴν ἀγορὴν προεῖπε Μιλησίοισι, ἐπεὰν αὐτὸς σημήνῃ, τότε πίνειν τε πάντας καὶ} \underline{\grc{κώμῳ}} \grc{χρᾶσθαι ἐς ἀλλήλους.}''\footnote{Herodotus, \emph{Histories} 1.21, ed. Godley.}
translated by Godley into ``he brought together into the marketplace all the food in the city, from private stores and his own, and told the men of Miletus all to \underline{drink and celebrate} [emphasis added] together when he gave the word,''\footnote{Herodotus, \emph{Histories} 1.21, trans. Godley.} 
Used in the context of a negotiator who offers a truce, then collects everything eatable and drinkable for a big public festivity doing an official saying. 
In Plato, Theaetetus 173d by Burnet, it reads as ``\grc{νόμους δὲ καὶ ψηφίσματα λεγόμενα ἢ γεγραμμένα οὔτε ὁρῶσιν οὔτε ἀκούουσι: σπουδαὶ δὲ ἑταιριῶν ἐπ᾽ ἀρχὰς καὶ σύνοδοι καὶ δεῖπνα καὶ σὺν αὐλητρίσι} \underline{\grc{κῶμοι}},''\footnote{Plato, \emph{Theaetetus} 173d, ed. Burnet.}
translated by Fowler into ``and the strivings of political clubs after public offices, and meetings, and banquets, and \underline{revelings} [emphasis added] with chorus girls''\footnote{Plato, \emph{Theaetetus} 173d, trans. Fowler.} 
Used in the context of a poem or so about the majesty of a great city with its revellings and beautiful flute girls.
In Plato, Republic 573d by Burnet, it reads as ``\grc{οἶμαι γὰρ τὸ μετὰ τοῦτο ἑορταὶ γίγνονται παρ᾽ αὐτοῖς καὶ} \underline{\grc{κῶμοι}} \grc{καὶ θάλειαι καὶ ἑταῖραι καὶ τὰ τοιαῦτα πάντα, ὧν ἂν Ἔρως τύραννος ἔνδον οἰκῶν διακυβερνᾷ τὰ τῆς ψυχῆς ἅπαντα,}''\footnote{Plato, \emph{Republic} 573d, ed. Burnet.}
translated by Shorey into ``for, I take it, next there are among them feasts and carousals and \underline{revelings} [emphasis added] and courtesans and all the doings of those whose souls are entirely swayed by the indwelling tyrant Eros,''\footnote{Plato, \emph{Republic} 573d, trans. Shorey.}
Used in the context of a fest to Eros with diverse celebrations, indwelling of a tyrant as demon possession together with, among others, temple-prostitutes. 
In Xenophon, Cyropaedia 7.5.25 by unknown, it reads as ``\underline{\grc{κωμάζει}} \grc{γὰρ ἡ πόλις πᾶσα τῇδε τῇ νυκτί,}''\footnote{Xenophon, \emph{Cyropaedia} 7.5.25, ed. unknown.}
translates partially by Miller into ``in view of the \underline{revelry} [emphasis added], it would not be at all surprising if the gates leading to the palace were open, for all the city is feasting this night,''\footnote{Xenophon, \emph{Cyropaedia} 7.5.25, trans. Miller.}
Used in context where there is an ongoing revelry in the palace, and the whole city is feasting all night. 
In total \grc{κῶμος} is a festivity that builds on making pacts jointly with feasting and carousal publicly where the whole city is involved. That is an occult public celebration of making a pact, or a covenant, that is statewide. Often sensuality and maybe even prostitution is common, especially with \emph{hetairas} temple prostitutes. \emph{Revel} defines as ``to dance, drink, sing, etc. at a party or in public, especially in a noisy way,''\cdfoot{revel}{2023-03-21} and \emph{festival} as ``a special day or period, usually in memory of a religious event, with its own social activities, food, or ceremonies.''\cdfoot{festival}{2023-03-21}
Found in Gal 5:21; Rom 13:13; 1~Pet. 4:3.
