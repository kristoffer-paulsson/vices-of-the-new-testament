\section{Notice of Use}

The intended use of this paper is to function as a dictionary of New Testament vices. It only includes the vices given in particular lists by Jesus and the apostles, the remaining vices are excluded. The entries are in alphabetical order according to the 
 English words used. 
 
 Each word or phrase found in the lists has its own article,
 which describes how it has been researched and how it has been comprehended. The vice entry has a carefully selected
 word reflecting particularly the sin instead of the usually commonly translated word found in dictionaries
 because a word can have multiple senses but only one or some senses may count as a vice. Each sin has its own conclusion
 with definitions from the Cambridge Dictionary (CD) to define it, an international English understanding is used for 
 people not having English as their mother tongue language. 
 
 Each entry begins with its English word, followed by 
 the Greek word and its Romanized transliteration for those not acquainted with ancient Greek. At the end of each
 article after the definition, the bible reference for each occurrence where the vice is declared as a sin is given.
 
The paper richly uses cross references by pointing to a said page using an arrow and page number, which looks like 
this: \entref{brutal}, as an example of ibidendum for \grc{ἀσέλγεια}.
Many articles use advanced quotations from TDNT in which the author references other Greek words that also are listed
in this very paper, in such cases, the same ibid. is used within brackets for easy follow-up internally to this paper.
 
 At the end of the paper, all Included Lists of Vices ~($\rightarrow$~\pageref{inc_list_vices}) are referenced with their respective bible passage together with the 
 lexeme of the Greek words for each contained vice.
 There is also a Greek Glossary ~($\rightarrow$~\pageref{grc_gloss}) for each Greek word and phrase pointing to the respective article for each comprehended sense. 
 Many Greek vices point to several understandings if the Greek word has a rich plethora of semantic values and each 
 understanding has its own entry but is sorted alphabetically.
 
 Also, before the end sections, there are categorized lists of vices based on the very similar consequences of grouped
 vices for each referenced passage.  
   
   
   
