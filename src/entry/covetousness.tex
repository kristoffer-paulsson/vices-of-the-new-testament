\item[Covetousness,]
\entlbl{covetousness}

\grc{φθόνος}
\index[grc]{φθονος@\grc{φθόνος}}
(\textit{phthonos}):
\newglossaryentry{phthonos}
{
    name=\grc{φθόνος},
    description={\entrefgls{covetousness}},
    sort=φθονος@\grc{φθόνος}
}
Both Thayer and Gingrich allow the use of \emph{envy}, but Liddell reaches deeper and denotes ``\emph{ill-will} or \emph{malice}, esp. \emph{envy} or \emph{jealousy} of the good fortune of others \ldots \emph{through envy}, \ldots better to \emph{be envied} than pitied! \ldots \emph{envy for}, \emph{jealousy of}, \ldots \emph{will grudge}, \emph{deny}, \ldots \emph{envy} or \emph{jealousy felt by} another, \ldots \emph{envyings}, \emph{jealousies}, \emph{heartburnings}, \ldots a \emph{cause for indignation}, a \emph{reproach}.'' The total understanding seems to be the envyings of mostly material goods and opportunities. Also, the denoted type of envy leads to malice. Therefore, it is substantial to use \emph{covetousness} which defines as ``a strong wish to have something, especially something that belongs to someone else.''\cdfoot{covetousness}{2023-03-27}
Found in Gal 5:21; Rom 1:29; Titus 3:3; 1~Pet 2:1.
