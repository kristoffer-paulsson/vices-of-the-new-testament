\item[Machiavellian,]
\entlbl{machiavellian}

\grc{αὐθάδης}
\index[grc]{αυθαδης@\grc{αὐθάδης}}
(\textit{authadēs}):
\newglossaryentry{authadēs}
{
    name=\grc{αὐθάδης},
    description={\entrefgls{machiavellian}},
    sort=αυθαδης@\grc{αὐθάδης}
}
According to Bauernfeind, the term denotes ``The critical judgement \ldots of the egocentric attitude, which as such necessarily leads to arrogance,'' further denotes ``In the two passages in which \emph{authades} occurs in the NT the reference is to human impulse violating obedience to the divine command. In both cases it is religious leaders who are exposed to this danger or succumb to it.''\bkfoot{\grc{αὐθάδης}}{1:508--9}{\tdntBauernfeind{}} Then Thayer, Liddell, Gingrich and Danker generally agree upon \emph{self-willed}, \emph{self-pleasing}, and \emph{arrogant}. L. mentions beyond that ``\emph{dogged}, \emph{stubborn}, \emph{contumacious}, \emph{presumptuous}, \ldots \emph{remorseless}, \emph{unfeeling}.'' Let's define five of those words mentioned by L.
1.~\emph{Egocentric} is defined as ``thinking only about yourself and what is good for you,''\cdfoot{egocentric}{2023-04-19}
2.~\emph{Dogged} is defined as ``very determined to continue doing something, or trying to do something, even when this is difficult or takes a long time,''\cdfoot{dogged}{2023-04-19}
3.~\emph{Contumacious} is defined as ``refusing to obey or respect the law in a way that shows contempt,''\cdfoot{contumacious}{2023-04-19}
4.~\emph{Presumptuous} is defined as ``A person who is presumptuous shows little respect for others by doing things they have no right to do,''\cdfoot{presumptuous}{2023-04-19}
5.~\emph{Unfeeling} is defined as ``not feeling sympathy for other people's suffering.''\cdfoot{unfeeling}{2023-04-19}
In the final end, this resembles what could be called in modern-day terms a psychopath, narcissist, or why not machiavellian. Those are extremely dangerous persons who only think about themselves which are not fit as a bishop for the house of God.
Found in Titus 1:7.
