\item[Abomination,]
\entlbl{abomination}

\grc{βδέλυγμα}
\index[grc]{βδελυγμα@\grc{βδέλυγμα}} 
(\textit{bdelygma}):
\newglossaryentry{bdelygma}
{
    name=\grc{βδέλυγμα},
    description={\entrefgls{abomination}},
    sort=βδελυγμα@\grc{βδέλυγμα}
}
Liddell, Thayer, and Gingrich agree on \emph{abomination}. 
In the NT this term is mostly used about the antichrist as the ``abomination of desolation,'' 
and about all the evil deeds of ``the great whore of Babylon (cf. Rev 17:5).'' 
The question then stands. What is the definition of abomination in such way that they are not welcome by God, in a practical
point of view in NT application of sinners? The Greek text says ``\grc{ὁ ποιῶν βδέλυγμα}'' which may translate 
``those doing abomination (NA28; my translation).''
Luke 16:15 mentions ``because the lofty in men is an abomination before God (NA28; my translation),'' which means 
in such case that the practice of the Pharisees, declaring themselves rightous before other people, is such an abomination.
Therefore practising selfrightousness to an extent similar to that of the Pharisees in the NT narative, should come close to 
what it means to do the abomination.
Found in Rev 21:27.
