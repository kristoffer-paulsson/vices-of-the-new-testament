\item[Shameful gain,]
\entlbl{shameful gain} 

\grc{αἰσχροκερδής}
\index[grc]{αισχροκερδης@\grc{αἰσχροκερδής}} 
(\textit{aischrokerdēs}):
\newglossaryentry{aischrokerdēs}
{
    name=\grc{αἰσχροκερδής},
    description={\entrefgls{shameful gain}},
    sort=αισχροκερδης@\grc{αἰσχροκερδής}
}
Stems from two words: \grc{αἰσχρός} \entref{shameful}, and \grc{κέρδος}, which according to Schlier means ```to gain,' `advantage,' `profit,' \ldots A derived sense is that of the `desire for gain of profit.' \ldots \emph{kērdos} is often used in the plur. (Hom.) in the sense of `crafty counsels,' `cunning' \ldots In the NT Titus 1:11 refers to the \emph{aischron kerdos} for the sake of which members of the community teach what they ought not.''\bkfoot{\grc{κέρδος}}{3:672--3}{\tdntSchlier{}} 
Further, Liddell states: \emph{sordidly greedy for gain}, and Thayer states: \emph{from eagerness for base gain}. Simply shameful gain.
Found in 1~Tim 3:8; Titus 1:7.
