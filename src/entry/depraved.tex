\item[Depraved,]
\entlbl{depraved}

\grc{ἀκάθαρτος}
\index[grc]{ακαθαρτος@\grc{ἀκάθαρτος}}
(\textit{akathartos}):
\newglossaryentry{akathartos}
{
    name=\grc{ἀκάθαρτος},
    description={\entrefgls{depraved}},
    sort=ακαθαρτος@\grc{ἀκάθαρτος}
}
In LXX, the term is about bio-hazards only. See \entref{depravity} for an explicit understanding. In Zech 13:2, there is a promise that God will remove the depraved spirit and the false prophets out of the land of Israel, also in the spiritual Heavenly Kingdom. Therefore, no depraved person (Eph 5:5) that is an idolater will enter. The unclean spirits mentioned in the NT 23 times, is one depraved spirit demon that enters through a vile lifestyle. According to Liddell, the cognate \grc{ἀκαθαρτίζομαι} means ``\emph{to be ceremonially unclean},'' then the term \grc{ἀκάθαρτος} denotes ``A. \emph{uncleansed}, \emph{foul}, \ldots of the body, \ldots of a woman, \ldots of ceremonial impurity, \ldots b. \emph{unpurified}, \ldots 2. \emph{morally unclean}, \emph{impure}, \ldots 3. of things, \emph{not purged away}, \emph{unpurged}, \ldots b. \emph{unpruned}, \ldots c. \emph{ceremonially unclean}, of food, \ldots d. \emph{not sifted}, \emph{containing impurities}, \ldots II. Act., \emph{not fit for cleansing}.'' 
Now the difference between impure and ceremonial unclean is the difference in how you treat fellow women (1~Pet 3:7).
Found in Eph 5:5.
