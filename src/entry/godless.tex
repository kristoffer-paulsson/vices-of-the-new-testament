\item[Godless,]
\entlbl{godless}

\grc{ἀσεβής}
\index[grc]{ασεβης@\grc{ἀσεβής}} 
(\textit{asebēs}):
\newglossaryentry{asebēs}
{
    name=\grc{ἀσεβής},
    description={\entrefgls{godless}},
    sort=ασεβης@\grc{ἀσεβής}
}
According to Foerster, for the pagans, irreverence meant ``[A.2]In Athenian trials for \emph{asebia} non-belief in the gods in which the \emph{polis} believes is called \emph{adikein} \ldots all private cults are to be forbidden under the law of \emph{asebia} \ldots As long as the ancient \emph{polis} endured \emph{asebia}, the failure to worship the city gods, was a breach of its order. Taking part in the national cult was \emph{eusebeia}, refusal to do so \emph{asebeia}. To the end of antiquity, then, an important part of \emph{eusebeia} or \emph{asebeia} was participation in the national cult or refusal to do this. \ldots As \emph{eusebeia} developed from reverence for the gods and the orders protected by them to worship of the gods, so \emph{asebeia} developed from a lack of reverence for the gods  to neglect of the cultus.''\bkfoot{\grc{ἀσεβής}}{7:186}{\tdntFoerster{}}
For the Jews, irreverence could be ``[B]With \emph{asebēs} there is never any more precise indication of obj. This does not mean that only God is the direct object of \emph{aseb-}. A first pt. to notice is that \emph{aseb-} never denotes a mere attitude but always action, conduct. Hence \emph{asebeia} can often be put in the plur. as issue of specific acts. If the obj. is indicated only with the verb \emph{asebeō}, this is the more surprising in that the group is not used only for cultic or particularly religious acts. A false witness speaks \emph{asebeia} \ldots \emph{asebēs} means one who is guilty in the judgment \ldots \emph{asebeō} means transgression of judicial directions.''\bksfoot{7:187}{Foerster}
\emph{Godless} defines as ``not having or believing in God or gods,'' even ``not showing belief in, or respect for, God.''\cdfoot{godless}{2023-04-25}
Found in 1~Tim 1:9.

