\item[Choked,]
\entlbl{choked}

\grc{πνικτός}
\index[grc]{πνικτος@\grc{πνικτός}}
(\textit{pniktos}):
\newglossaryentry{pniktos}
{
    name=\grc{πνικτός},
    description={\entrefgls{choked}},
    sort=πνικτος@\grc{πνικτός}
}
In current verses, the choked stands in direct context with \entref{blood}. Both Bietenhard\bkfoot{\grc{πνικτός}}{6:457}{\tdntBietenhard{}}, Thayer, and Gingrich agree with what G. denotes: ``\emph{strangled}, \emph{choked to death} of animals killed for food without having the blood drained from them'' The connection between blood and the choked is that blood can be used for more than eating, like ceremonial purposes, therefore, the choked clarifies the eating part of blood even stronger.
Found in Acts 15:20, 29, 21:25.
