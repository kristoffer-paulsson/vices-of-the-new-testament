\item[Profane,]
\entlbl{profane} 

\grc{βέβηλος}
\index[grc]{βεβηλος@\grc{βέβηλος}} 
(\textit{bebēlos}):
\newglossaryentry{bebēlos}
{
    name=\grc{βέβηλος},
    description={\entrefgls{profane}},
    sort=βεβηλος@\grc{βέβηλος}
}
According to Hauck, it means ``the place which may be entered by anyone \ldots `accessible.' It corresponds exactly to the Lat. \emph{profanus}. \ldots it is used of persons in the sense of `unsanctified' or `profane.' \ldots as applied to persons, \emph{bebēlos} in Heb 12:16 (alongside \emph{pornos} [i. \entref{male-sex-buyer} ii. \entref{male-prostitute}]) \ldots denotes profane men who are far from God; their unholiness includes ethical deficiency in accordance with the NT approach.''\bkfoot{\grc{βέβηλος}}{1:604--5}{\tdntHauck{}}
Profane is the use of the world for a vessel that is supposed to be clean and used for holy purposes, therefore it's not allowed to use it for worldly purposes but for holy use in the world.
\emph{Worldly} defines as ``experienced in the ways of the world,'' and as ``relating to or consisting of physical things and ordinary life rather than spiritual things.''\cdfoot{worldly}{2023-03-08}
Found in 1~Tim 1:9; Heb 12:16.
