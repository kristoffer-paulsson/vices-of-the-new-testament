\item[Prostitution,]
\entlbl{prostitution} 

\grc{πορνεία}
\index[grc]{πορνεια@\grc{πορνεία}} 
(\textit{porneia}):
\newglossaryentry{porneia}
{
    name=\grc{πορνεία},
    description={i. \entrefgls{prostitution} ii. \entrefgls{sex-purchase}},
    sort=πορνεια@\grc{πορνεία}
}
According to Hauck/Schulz, ``\emph{pornē} from \emph{pernēmi} `to sell,' esp. of slaves, means lit. `harlot for hire,' `prostitute'; Gk. harlots were usually bought slaves \ldots \emph{pornos} [\entref{male-sex-buyer}] \ldots `whoremonger' who has intercourse with prostitutes then specifically one who lets himself be used for money, `male prostitute' \ldots \emph{porneia} \ldots rare in class Gk., `fornication' \ldots \emph{moicheuō} [\entref{adultery}] is narrower than \emph{porneia} and refers solely to adultery.''\bkfoot{\grc{πορνεία}}{6:580}{Hauck/Schulz}
According to Liddell the following word \grc{πορνεία} means \emph{prostitution}, 
then: examining the cognates found in L. gives a broader understanding of what the concept of \grc{πορν-} is,
\grc{πορνεῖον} \emph{brothel}, 
\grc{πορνεύω} ``\emph{prostitute}, mostly in Pass., of a woman, \emph{prostitute herself}, \emph{be or become a prostitute},'' 
\grc{πόρνη} ``\emph{harlot}, \emph{prostitute},'' 
\grc{πορνικός} ``\emph{of or for harlots}, \ldots the tax \emph{paid by brothel-keepers},'' 
\grc{πορνοβοσκέω} ``\emph{keep a brothel},'' 
\grc{πορνοβοσκία} ``\emph{trade of a brothel-keeper},'' 
\grc{πορνοβοσκός} \emph{brothel-keeper}, 
\grc{πορνοφίλας} or \grc{πορνο-φίλης} ``\emph{loving-harlots},'' 
\grc{πορνο-διδάσκαλος} ``\emph{teacher of fornication},'' 
\grc{πορνο-γέννητος} ``\emph{born of harlots},'' 
\grc{πορνο-γράφος} ``\emph{wiritng of harlots},'' 
\grc{πορνο-κοπέω} ``\emph{to be a whoremonger},'' 
\grc{πορνο-κοπία} \emph{whoremongering}, 
\grc{πορνο-κόπος} ``\emph{one who has commerce with prostitutes}, \emph{fornicator},'' 
\grc{πορνο-μανής} ``\emph{mad after prostitutes},'' 
\grc{πόρνος} ``\emph{catamite}, \emph{sodomite}, \emph{fornicator},'' 
\grc{πορνο-τελώνης} ``\emph{farmer of the} [brothel], \emph{tax-gatherer}.'' 
A lot of bible translations use the word \emph{fornication} which stems from the Latin. Lewis \& Short says: fornix ``\emph{a brothel},  \emph{bagnio}, \emph{stew}, situated in underground vaults,''
then: examining the cognates found in L. \& S. gives a broader understanding of what the concept of forni- is,
fornĭcātĭo ``\emph{whoredom}, \emph{fornication},'' 
fornĭcātor \emph{fornicator}, 
fornĭcor ``\emph{to commit whoredom} or \emph{fornication},'' 
fornĭcātrix ``\emph{a fornicatress}, prostitute.'' 
It is clear that \emph{fornication} comes from \emph{prostitution}, and everything related to \grc{πορν-} is brothel activity and nothing else. Therefore it should be obvious that \grc{πορνεία} is the activity of a man paying for the performance of prostitutive acts and nothing else. Prostitution has come to be a derogatory term for those not practicing brothel activity but those that refuse religious \emph{sexual asceticism} and similar. \emph{Prostitution} defines as ``the business of having sex for money.''\cdfoot{prostitution}{2023-03-10} (It seems like CD has removed the act of a man buying a prostitute from the dictionary.)
Found in Gal 5:19; Matt 15:19; Mark 7:21; Eph 5:3; 2~Cor 12:21; Col 3:5; Acts 15:20, 29, 21:25; Rev 9:21.
