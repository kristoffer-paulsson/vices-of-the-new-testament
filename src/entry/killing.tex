\item[Killing,]
\entlbl{killing} 

\grc{φόνος}
\index[grc]{φονος@\grc{φόνος}} 
(\textit{phonos}):
\newglossaryentry{phonos}
{
    name=\grc{φόνος},
    description={\entrefgls{killing}},
    sort=φονος@\grc{φόνος}
}
According to Liddell, there is a long denotation, but Thayer, Gingrich, and Danker may together shortly denote \emph{murder}, \emph{homicide}, \emph{slaughter}, \emph{killing}, and \emph{execution}. Then L. shortly denotes ``\emph{murder}, \emph{slaughter}, \ldots exact vengeance for the \emph{killing} \ldots \emph{killing} or not-\emph{killing}, \ldots the \emph{murder} of \ldots \emph{slaughter} of Greeks, \ldots in law, \emph{murder}, \emph{homicide}, \ldots \emph{death as a punishment}, \ldots \emph{blood when shed}, \emph{gore}, \ldots of a sacrifice, \ldots rarely in Prose of \emph{blood}, \ldots \emph{corpse}, \ldots \emph{rascal that deserves death}, \emph{gallowsbird}, a Dorian phrase, \ldots of the agent or instrument of slaughter, \ldots to be \emph{a death} to heroes, \ldots of poison.'' It seems that partaking in the death penalty as an executioner could count as the sin of killing. Also, may it be prohibited to carry out a lethal action as of non-death penalty. However, whether causing a killing by self-defense of home (Luke 11:21), wife (Eph 5:25), or children and assaulting a burglar in one's own home (Matt 24:43; Exod 22:2) may be said to count as exceptions thereof. Matt 15:19 and Mark 7:21, which mentions an inner evil contemplation of killing in one's heart, also counts as sin.
Found in Matt 15:19; Mark 7:21; Rom 1:29; 1~Tim 1:9; Rev 9:21.
