\item[Betrayer,]
\entlbl{betrayer}

\grc{προδότης}
\index[grc]{προδοτης@\grc{προδότης}}
(\textit{prodotēs}):
\newglossaryentry{prodotēs}
{
    name=\grc{προδότης},
    description={\entrefgls{betrayer}},
    sort=προδοτης@\grc{προδότης}
}
All Liddell, Thayer, and Gingrich mention \emph{betrayer} together with \emph{traitor}. 
L. further mentions ``\ldots \emph{traitor} to his oaths, \ldots \emph{one who abandons in danger}.''
Only used twice in the NT in Luke 6:16 of Judas Iscariot, who betrayed Jesus, and in Acts 7:52, where Stephen is preaching against the religious leaders and their betrayal of God’s messengers, the prophets. The difference is the biblical use as active betrayal, while secular use may be abandonment out of cowardice.
\emph{Betrayer} is ``a person who is not loyal to their country or to another person, often doing something harmful such as giving information to an enemy.''\cdfoot{betrayer}{2023-04-01}
Found in 2~Tim 3:4.
