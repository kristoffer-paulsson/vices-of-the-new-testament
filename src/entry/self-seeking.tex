\item[Self-seeking,]
\entlbl{self-seeking} 

\grc{ἐριθεία}
\index[grc]{εριθεια@\grc{ἐριθεία}} 
(\textit{eritheia}):
\newglossaryentry{eritheia}
{
    name=\grc{ἐριθεία},
    description={\entrefgls{self-seeking}},
    sort=εριθεια@\grc{ἐριθεία}
}
According to Büchsel, said term denotes ``\emph{eritheia} comes from \emph{eritheuō}, `to work as a day-labourer,' `to conduct oneself as such,' `to work for daily hire,' and this again comes from \emph{erithos}, a `day-labourer.' \emph{eritheia} thus means the `work,' then the `manner, attitude or disposition of the day-labourer.' \ldots here are those who procure office by illegal manipulation, and therefore \emph{eritheia} is their attitude, i.e., not so much \emph{ambitus} as a action, but the personal manner connected with it \ldots The adj. is used in the same sense in the civic oath of the Italians: \ldots `I will not on any pretext bring a charge of failure to keep civic law against any citizen for personal reasons.' \emph{erithiotan}, the crucial point of the oath, defines such charges as unobjective and self-seeking. \ldots He thus demands that leaders should be non-contentious and without personal ambition. \ldots Ezek 23:5 \ldots v11 \emph{eritheia}(LXX: \emph{epithesis}), of a harlot who offers herself to a man or who entices him. \emph{eritheia} is thus the attitude of self-seekers, harlots etc., i.e., those who demeaning themselves and their cause, are busy and active in their own interest, seeking their own gain or advantage. \ldots For many it probably had no more than the general sense of baseness, self-interest, ambition, contention. \ldots For this reason, it is best to understand \emph{eritheia} as `base self-seeking,' or simply as `baseness,' the nature of those who cannot lift their gaze to higher things.''
\emph{Baseness} defines as ``a lack of any honor or morals,''\cdfoot{besness}{2023-03-16} also \emph{self-seeking} as ``interested in your own advantage in everything that you do.''\cdfoot{self-seeking}{2023-03-16}
Found in Gal 5:20; 2~Cor 12:20.
