\item[Sectarianism,]
\entlbl{sectarianism} 

\grc{αἵρεσις}
\index[grc]{αιρεσις@\grc{αἵρεσις}} 
(\textit{hairesis}):
\newglossaryentry{hairesis}
{
    name=\grc{αἵρεσις},
    description={\entrefgls{sectarianism}},
    sort=αιρεσις@\grc{αἵρεσις}
}
Danker describes it as a ```choice of association based on shared principles or beliefs', ordinarily of a subgroup with views or beliefs that deviate in certain respects from those of the larger membership party, faction.'' Usually, there are religious or schismatic sects that rise in heresy and cause divisions in the Body of Christ contrary to the gospel. Schlier adds that ``It thus comes to be the \emph{hairesis} (teaching) of a particular \emph{hairesis} (school). \ldots For the concept of such a fellowship --- as well as \ldots we also have \ldots --- the following aspects are important: the gathering of the \emph{hairesis} from a comprehensive society and therefore its delimitation from other schools; the self-chosen authority of a teacher; the relatively authoritarian and relatively disputable doctrine''\bkfoot{\grc{αἵρεσις}}{1:181}{\tdntSchlier{}} compared to the Word of God. 
Thayer inclines that the verb form \grc{αἱρέω} denotes ``\emph{act of taking}, \emph{capture}: \ldots \emph{the storming of a city}'' From that, it's conclusive that unnecessary divisions in a group are about making a coup based on disputably misunderstood teachings. \emph{Sectarianism} defines as ``very strong support for the religious or political group that you are a member of, which can cause problems between different groups.''\cdfoot{sectarianism}{2023-03-05}
Found in Gal 5:20.



